\documentclass[a4page]{article}
% \usepackage{biblatex}
\usepackage[a4paper,
            bindingoffset=0.2in,
            left=1in,
            right=1in,
            top=1in,
            bottom=1in,
            footskip=.25in]{geometry}
\usepackage{amsmath, amsfonts, amsthm , mathtools , bbold, float}

\usepackage{tikz}
\usepackage{tikz-cd}
\usepackage{mathpartir}
\usepackage[citecolor=blue,linkcolor=blue,urlcolor=blue,colorlinks]{hyperref}
\usepackage[nameinlink,capitalize,noabbrev]{cleveref}
\usepackage{quiver}
\usepackage{xspace}
\usepackage{xcolor}
\usepackage{bbm}
\usepackage{breakurl}
\usepackage{bbding}
\hypersetup{pdfborder={0 0 100}}
\usepackage{lipsum}
%\usepackage{biblatex}
%\bibliographystyle{unsrt}
\usepackage[backend=biber,style=alphabetic,sorting=nty,giveninits]{biblatex} % Bibliography management
\addbibresource{refs.bib}



\usepackage{textcomp}
\usepackage{xspace}
\usepackage{url}
\usepackage{multirow}
\usepackage{hhline}
\usepackage{pifont}
\usetikzlibrary{arrows,matrix,decorations.pathmorphing,
  decorations.markings, calc, backgrounds}
\usepackage{mathpartir}
\usepackage{microtype}
%\DeclareMathOperator*{\bigsmile}{\raisebox{-0.ex}{\scalebox{1.5}{$\smile$}}}

\DisableLigatures[-]{family=tt*}

\definecolor{darkgreen}{rgb}{.30,.70,.60}

\newtheorem{remark}{Remark}
\newtheorem{lemma}{Lemma}
\newtheorem{definition}{Definition}
\newtheorem{example}{Example}
\newtheorem{proposition}{Proposition}
\newtheorem{theorem}{Theorem}
%\newtheorem{theoremCheck}[theorem]{Theorem${}^{\text{\color{darkgreen}{\Checkmark}}}$}
\newtheorem{corollary}{Corollary}
\newtheorem{conjecture}{Conjecture}


\newcommand{\push}[1]{{\ensuremath{\textnormal{\textsf{push}}}(#1)}}
\newcommand{\pushl}[1]{{\ensuremath{\textnormal{\textsf{push}}_{\textnormal{\textsf{l}}}}(#1)}}
\newcommand{\pushr}[1]{{\ensuremath{\textnormal{\textsf{push}}_{\textnormal{\textsf{r}}}}(#1)}}
\newcommand{\pushlr}{{\ensuremath{\textnormal{\textsf{push}}_{\textnormal{\textsf{lr}}}}}}
\newcommand{\smashinc}[2]{{\ensuremath{\langle #1,#2 \rangle}}}
\newcommand{\smashincalt}[3]{{\ensuremath{\langle #1,#2,#3 \rangle}}}

\newcommand{\incl}[1]{\textnormal{\ensuremath{\mathsf{incl}_{#1}}}}
\newcommand{\fin}[1]{\textnormal{\ensuremath{\mathsf{Fin}({#1})}}}


%% Configure appearance of system names here
\newcommand{\teletype}[1]{\ensuremath{\mathtt{#1}}}
\newcommand{\systemname}[1]{\teletype{\color{darkgray}#1}\xspace}

%% Systems referred to in the paper
\newcommand{\cubicaltt}{\systemname{cubicaltt}}
\newcommand{\cooltt}{\systemname{cooltt}}
\newcommand{\Idris}{\systemname{Idris}}
\newcommand{\Haskell}{\systemname{Haskell}}
\newcommand{\Lean}{\systemname{Lean}}

\definecolor{Revolutionary}{RGB}{232,70,68}
\newcommand{\redtt}{\textbf{\teletype{{\color{Revolutionary}red}tt}}}

\newcommand{\eg}{{e.g.}}
\newcommand{\ie}{{i.e.}}
\newcommand{\cf}{{cf.}}

%% Type theory macros
\newcommand{\su}[2]{#1/#2}
\newcommand{\subst}[2]{(\su #1 #2)}
\newcommand{\esubst}[3]{(#1, \su #2 #3)}
\newcommand{\substnop}[2]{{#2}\, / \,{#1}}

%% \usepackage{MnSymbol}
\usepackage{newunicodechar}
\newunicodechar{≃}{\ensuremath{\simeq}}
\newunicodechar{≅}{\ensuremath{\mathsfg}}
\newunicodechar{∈}{\ensuremath{\in}}
\newunicodechar{⁻}{\ensuremath{^{-}}}
\newunicodechar{ᴹ}{\ensuremath{_{M}}}
\newunicodechar{ᴺ}{\ensuremath{_{N}}}
\newunicodechar{≡}{\ensuremath{\equiv}}
\newunicodechar{λ}{\ensuremath{\lambda}}
\newunicodechar{⊎}{\ensuremath{\uplus}}
\newunicodechar{∷}{\ensuremath{::}}
\newunicodechar{ℓ}{\ensuremath{\ell}}
\newunicodechar{ᵢ}{\ensuremath{_i}}
\newunicodechar{⟨}{\ensuremath{\langle}}
\newunicodechar{⟩}{\ensuremath{\rangle}}
\newunicodechar{α}{\ensuremath{\alpha}}
\newunicodechar{β}{\ensuremath{\beta}}
\newunicodechar{θ}{\ensuremath{\theta}}
\newunicodechar{φ}{\ensuremath{\varphi}}
\newunicodechar{ψ}{\ensuremath{\psi}}
\newunicodechar{η}{\ensuremath{\eta}}
\newunicodechar{ε}{\ensuremath{\varepsilon}}
\newunicodechar{ι}{\ensuremath{\iota}}
\newunicodechar{Σ}{\ensuremath{\Sigma}}
\newunicodechar{σ}{\ensuremath{\sigma}}
\newunicodechar{∀}{\ensuremath{\forall}}
\newunicodechar{ℕ}{\ensuremath{\mathbb{N}}}
\newunicodechar{→}{\ensuremath{\to}}
\newunicodechar{⊎}{\ensuremath{\uplus}}
\newunicodechar{⋆}{\ensuremath{*}}
\newunicodechar{¬}{\ensuremath{\lnot}}
\newunicodechar{Θ}{\ensuremath{\theta}}
\newunicodechar{ᴸ}{\ensuremath{{}^{\color{black}\leftarrow}}}  % hax
\newunicodechar{ᴿ}{\ensuremath{{}^{\color{black}\rightarrow}}} % hax
\newunicodechar{∧}{\ensuremath{\wedge}}
\newunicodechar{∨}{\ensuremath{\vee}}
\newunicodechar{∼}{\ensuremath{\sim}}
\newunicodechar{≢}{\ensuremath{\nequiv}}
\newunicodechar{Π}{\ensuremath{\Pi}}
\newunicodechar{ℤ}{\ensuremath{\mathbb{Z}}}
\newunicodechar{∥}{\ensuremath{\parallel}}
\newunicodechar{∣}{\ensuremath{\mid}}
\newunicodechar{ℚ}{\ensuremath{\mathbb{Q}}}
\newunicodechar{₊}{\ensuremath{_+}}
\newunicodechar{∗}{\ensuremath{\ast}}
\newunicodechar{₁}{\ensuremath{_1}}
\newunicodechar{₂}{\ensuremath{_2}}
\newunicodechar{⊥}{\ensuremath{\bot}}
\newunicodechar{∘}{\ensuremath{\circ}}
\newunicodechar{ρ}{\ensuremath{\rho}}
\newunicodechar{↦}{\ensuremath{\mapsto}}
\newunicodechar{₀}{\ensuremath{_0}}
\newunicodechar{∙}{\ensuremath{\boldsymbol{\cdot}}}
\newunicodechar{Ω}{\ensuremath{\Omega}}
\newunicodechar{§}{\ensuremath{\mathbb{S}}}
\newunicodechar{𝟙}{\ensuremath{\mathbb{1}}}
\newunicodechar{×}{\ensuremath{\times}}

%Ugly hacks for torus, klein and RP2
\newunicodechar{□}{\mathsf{\ensuremath{\square}}}
\newunicodechar{α}{\textsf{\ensuremath{{\mathbb{T}^2}}}} %Torus

\newunicodechar{β}{\textsf{\ensuremath{{\mathbb{K}^2}}}} %Klein

\newunicodechar{γ}{\textsf{\ensuremath{{\mathbb{R}P^2}}}} %RP^2


% Hacks for ex computation
\newunicodechar{ₕ}{\textsf{\ensuremath{_h}}}
\newunicodechar{ϕ}{\textsf{\ensuremath{\phi}}}
\newunicodechar{ζ}{\anum{3}\;\;\;\;\;\;\;\;\;}

\newcommand{\winding}[1]{\textsf{winding}({#1})}
\newcommand{\Type}{\textsf{Type}}
\newcommand{\ptrunc}[1]{\textsf{∥}\,{#1}\,\textsf{∥}}
\newcommand{\tyProduct}[2]{{#1}\,\AgdaOperator{\AgdaTextsftion{×}}\,{#2}}
\newcommand{\tySigma}[3]{\textsf{Σ[}\,{#1}\,\textsf{∈}\,{#2}\,\textsf{]}\,{#3}}
\newcommand{\tyMaybe}[1]{\data{Maybe}\,{#1}}
\newcommand{\tyNat}{\data{ℕ}}
\newcommand{\tyPath}[2]{{#1}\,\textsf{≡}\,{#2}}
\newcommand{\tyPathP}[4]{\primty{PathP}\,(\symb{λ}\,{#1} \to {#2})\,{#3}\,{#4}}
\newcommand{\tyEquiv}[2]{{#1}\,\textsf{≃}\,{#2}}
\newcommand{\tySum}[2]{{#1}\,\textsf{⊎}\,{#2}}
\newcommand{\tyNot}[1]{\textsf{¬}\,{#1}}
\newcommand{\tyStructEq}[3]{{#1}\,\textsf{≃[}\,{#2}\,\textsf{]}\,{#3}}



\newlength{\LETTERheight}
\AtBeginDocument{\settoheight{\LETTERheight}{I}}
\newcommand*{\longleadsto}[1]{\ \raisebox{0.24\LETTERheight}{\tikz \draw [->,
line join=round,
decorate, decoration={
    zigzag,
    segment length=4,
    amplitude=.9,
    post=lineto,
    post length=2pt
}] (0,0) -- (#1,0);}\ }

\newcommand{\cmark}{\ding{51}}%
\newcommand{\xmark}{\ding{55}}%
\newcommand{\equivDef}{\overset{\scriptscriptstyle \textsf{def}}{\equiv}}
\newcommand{\trivGrp}{\textsf{𝟙}}
\newcommand{\bZ}{\textsf{ℤ}}
\newcommand{\bN}{\textsf{$\mathbb{N}$}}
\newcommand{\sphere}[1]{\textsf{§}^{#1}}
\newcommand{\klein}{\textsf{$\mathbb{K}^{\,2}$}}
\newcommand{\cohom}[2]{H^{#1}\!\left({#2}\right)}
\newcommand{\funspace}[2]{\left({{#2} \to \textsf{K}_{#1}}\right)}
\newcommand{\funspaceP}[2]{{{#2} \to \textsf{K}_{#1}}}
\newcommand{\torus}{\textsf{$\mathbb{T}^{\,2}$}}
\newcommand{\RP}{\textsf{$\mathbb{R}P^2$}}
\newcommand{\RPinf}{\textnormal{\ensuremath{\mathbb{R}P^\infty}}}
\newcommand{\CP}{\textsf{$\mathbb{C}P^2$}}
\newcommand{\refl}{\textsf{refl}}
\newcommand{\inr}[1]{\mathsf{inr}\,{#1}}
\newcommand{\inl}[1]{\mathsf{inl}\,{#1}}
\newcommand{\trunc}[1]{\mathsf{∣}\,#1\,\mathsf{∣}}
\newcommand{\truncT}[2]{\textsf{∥}\,#2\,\textsf{∥}_{#1}}
\newcommand{\Loop}{\mathsf{loop}}
\newcommand{\base}{\mathsf{base}}
\newcommand{\north}{\mathsf{north}}
\newcommand{\south}{\mathsf{south}}
\newcommand{\Code}{\textsf{Code}}
\newcommand{\merid}[1]{\mathsf{merid}\;{#1}}
\newcommand{\transport}[2]{\textsf{transport}\;(#1)\;#2}
\newcommand{\ap}[2]{\textnormal{\textsf{ap}}_{#1}{(#2)}}
\newcommand{\app}[3]{\textnormal{\textsf{ap}$^2$}_{#1}\;{#2}\;{#3}}
\newcommand{\tmPair}[2]{{#1}\,\mathsf{,}\,{#2}}
\newcommand{\tmTrunc}[1]{\mathsf{[}\,{#1}\,\mathsf{]}}
\newcommand{\tmNil}{\mathsf{[]}}
\newcommand{\tmCons}[2]{{#1}\;\mathsf{∷}\;{#2}}

\newcommand{\cupprodop}{\textsf{$\smile$}}
\newcommand{\cupprodkop}{\textsf{$\smile_k$}}
\newcommand{\cupprodk}[2]{{#1}\;\cupprodkop\;{#2}}
\newcommand{\cupprod}[2]{{#1}\;\cupprodop\;{#2}}

%% Comment macro colors
\definecolor{dkblue}{rgb}{0,0.1,0.5}
\definecolor{lightblue}{rgb}{0,0.5,0.5}
\definecolor{dkgreen}{rgb}{0,0.6,0}
\definecolor{dkbrown}{rgb}{0.4,0,0}
\definecolor{dkviolet}{rgb}{0.6,0,0.8}

%% SUBMISSION: turn off all comments
%\newcommand{\mycomment}[3]{}
\newcommand{\mycomment}[3]{\textcolor{#1}{[#2#3]}}
\newcommand{\todo}[1]{\mycomment{red}{TODO: }{#1}}

\newcommand{\sq}[1]{\textnormal{\ensuremath{\mathsf{Sq}^{#1}}}}
\newcommand{\sqind}[2]{\textnormal{\ensuremath{\mathsf{Sq}_{#2}^{#1}}}}

\DeclareMathOperator*{\bigast}{\raisebox{-0.6ex}{\scalebox{2.5}{$\ast$}}}
\DeclareMathOperator*{\bigsmile}{\raisebox{-0.ex}{\scalebox{1.5}{$\smile$}}}

\newcommand{\negpath}{\textnormal{\textsf{neg-path}}}
\newcommand{\totSq}{\textnormal{\ensuremath{\widehat{\mathsf{Sq}}}}}

\newcommand{\cwskel}{\textnormal{\ensuremath{\mathsf{CW}^{\mathsf{skel}}}}}
\newcommand{\cwskelinf}{\textnormal{\ensuremath{\mathsf{CW}_\infty^{\mathsf{skel}}}}}
\newcommand{\hskel}[1]{\textnormal{\ensuremath{H^{\mathsf{skel}}_{#1}}}}
\newcommand{\hskelinf}[1]{\textnormal{\ensuremath{H^{\mathsf{skel}_\infty}_{#1}}}}
\newcommand{\abgrp}{\textnormal{\ensuremath{\mathsf{AbGrp}}}}


% To change indentation of code:
% \setlength{\mathindent}{5pt}

% \bibliographystyle{plainurl}% the mandatory bibstyle

\title{Cellular Homology and the Cellular Approximation Theorem in HoTT}

\author{
  An author\\
  \footnotesize{SU}\\
  \footnotesize{\texttt{hoho@math.su.se}}
  \and
  Another author\\
  \footnotesize{SU}\\
  \footnotesize{\texttt{haha@math.su.se}}
  \and
  A third author\\
  \footnotesize{SU}\\
  \footnotesize{\texttt{hehe@math.su.se}}
}
\date{}

\begin{document}

\maketitle

In~\cite{BuchholtzFavonia18}, Buchholtz and Favonia develops a theory
of cellular cohomology in HoTT. This is done in two steps. Step 1 is
the construction of cohomology groups of (the skeleton of) a CW
complex using the usual definition via cochain complexes.  Step 2 is
the construction of an isomorphism between their definition of
cohomology the more well-established definition in terms of
Eilenberg-MacLane spaces (see~\cite{LicataFinster14}). Step 2 allows
the authors to infer all e.g. functoriality and the Eilenberg-Steenrod
axioms of their cohomology theory. Step 2 is, however, not as readily
available when developing cellular homology---despite homology having
been studied in HoTT~\cite{graham18,christensen2020hurewicz}, it
generally considered harder to work with than cohomology. \todo{blabla bla rewrite please...}

We revisit \cite{BuchholtzFavonia18} definition of cellular chain
complexes and develop a functorial homology theory, not via reduction
to another more well-studied homology theory, but via cellular
maps. This crucially relies on the so called \emph{cellular
  approximation theorem} which we prove (a constructive analogue of) in HoTT.

Let $\bN^{-} := \bN \cup \{-1\}$ and $\mathsf{Fin}(n)$ denote the $n$-element set. We will use the following definition of CW
\begin{definition}[CW skeleta]
  A \textbf{CW-skeleton} is an infinite sequence of types and maps
  \[X_{-1} \xrightarrow{\incl{-1}} X_0 \xrightarrow{\incl{0}} X_1 \xrightarrow{\incl{1}} \dots \]
  equppied with a function $c : \bN^- \to \bN$ and a set of attaching maps $\alpha_i : S^i \times \fin{c_i} \to X_i$ such that $X_{-1}$ is empty and the following square is a pushout.
  \[
\begin{tikzcd}[ampersand replacement=\&]
	{S^{i} \times \fin{c(i)}} \&\& \fin{c(i)} \\
	{X_i} \&\& {X_{i+1}}
	\arrow["{\mathsf{snd}}", from=1-1, to=1-3]
	\arrow["{\alpha_{i}}"', from=1-1, to=2-1]
	\arrow[from=2-1, to=2-3]
	\arrow[from=1-3, to=2-3]
	\arrow["\lrcorner"{anchor=center, pos=0.125, rotate=180}, draw=none, from=2-3, to=1-1]
\end{tikzcd}
\]
A CW-skeleton is said to be \textbf{finite (of dimension $n$)} if $\incl{m}$ is an equivalence for all $m \geq n$.
\end{definition}
This definition is slightly different from the recursive definition employed in (the formalisation of)~\cite{BuchholtzFavonia18}. Its usefulness is two-fold: first, it allows us to also define infinite dimensional CW complexes, such as e.g. $\RPinf$. Second, it allows us to extract the $n$-skeleton, $X_n$, of a CW-skeleton $X_{(-)}$ directly without having to rely on auxilliary functions.
\begin{definition}[CW complexes]\todo{Not sure if all these defs are needed...}
  A type $A$ is said to be a \textbf{CW complex} if there merely exists some CW skeleton $(X_i)$ such that $A$ is equivalent to $X_\infty$, where $X_\infty$ is the sequential colimit of the sequence $(X_i)$. We say that $A$ is of \textbf{dimension $n$} if $(X_i)$ is finite of dimension $n$ and that $A$ is \textbf{finite} if there merely exists some $n$ such that $A$ is of dimension $n$.
\end{definition}

Let $\bZ[n]$ denote the free abelian group with $n-1$ generators with $\bZ[0]$ defined to be the trivial group. Buchholtz and Favonia~\cite{BuchholtzFavonia18} showed how to construct the chain complex
$\dots \xrightarrow{\partial_1} \bZ[c_{1}] \xrightarrow{\partial_0} \bZ[c_{0}]$ associated to a CW-skeleton $(X,\mathsf{incl},c,\alpha)$.

\todo{write here}

\begin{definition}
  Let $(X_i)$ and $(Y_i)$ be skeleta. A \textbf{cellular map}, which we simply denote $f : (X_i) \to (Y_i)$, consists of a famly of functions $f_i : X_i \to Y_i$ for each $i \geq -1$ making the following diagram commute
  \[
\begin{tikzcd}[ampersand replacement=\&]
	{X_{i+1}} \&\& {Y_{i+1}} \\
	{X_i} \&\& {Y_i}
	\arrow["{f_{i+1}}", from=1-1, to=1-3]
	\arrow["{f_{i}}"', from=2-1, to=2-3]
	\arrow[hook, from=2-1, to=1-1]
	\arrow[hook, from=2-3, to=1-3]
\end{tikzcd}
\]
\end{definition}

%% Given a CW skeleton $(X_i)$, let $(X_i^{(n)})$ denote the subcomplex obtained by setting $X_i^{(n)} = \begin{cases} X_i & \text{ if $x \leq n$} \\ X_n &\text{ otherwise} \end{cases}$.
\begin{theorem}[Cellular approximation]
  Let $(X_i)$ and $(Y_i)$ be CW skeleta and $f : X_{\infty} \to Y_{\infty}$ a map between their colimits. For any fixed $n \geq -1$, 
\end{theorem}
The existence of cellular map is what gives rise to the functoriality of $H_n(-)$, as made clear by the following theorem.
\begin{proposition}
  For every cellular map $f : (X_i) \to (Y_i)$, there is a homomorphism $f_* : H^{\mathsf{sk}}_n(X) \to H^{\mathsf{sk}}_n(Y)$. Furthermore, this assignment is functorial.
\end{proposition}
\begin{proof}
\end{proof}


\printbibliography


\end{document}
