\documentclass[a4page]{article}
% \usepackage{biblatex}
\usepackage[a4paper,
            bindingoffset=0.2in,
            left=1in,
            right=1in,
            top=1in,
            bottom=1in,
            footskip=.25in]{geometry}
\usepackage{amsmath, amsfonts, amsthm , mathtools , bbold, float}

\usepackage{tikz}
\usepackage{tikz-cd}
\usepackage{mathpartir}
\usepackage[citecolor=blue,linkcolor=blue,urlcolor=blue,colorlinks]{hyperref}
\usepackage[nameinlink,capitalize,noabbrev]{cleveref}
\usepackage{quiver}
\usepackage{xspace}
\usepackage{xcolor}
\usepackage{bbm}
\usepackage{breakurl}
\usepackage{bbding}
\hypersetup{pdfborder={0 0 100}}
\usepackage{lipsum}
%\usepackage{biblatex}
%\bibliographystyle{unsrt}
\usepackage[backend=biber,style=alphabetic,sorting=nty,giveninits]{biblatex} % Bibliography management
\addbibresource{refs.bib}

\usepackage{textcomp}
\usepackage{xspace}
\usepackage{url}
\usepackage{multirow}
\usepackage{hhline}
\usepackage{pifont}
\usetikzlibrary{arrows,matrix,decorations.pathmorphing,
  decorations.markings, calc, backgrounds}
\usepackage{mathpartir}
\usepackage{microtype}

\DisableLigatures[-]{family=tt*}

\definecolor{darkgreen}{rgb}{.30,.70,.60}

\newtheorem{remark}{Remark}
\newtheorem{lemma}{Lemma}
\newtheorem{definition}{Definition}
\newtheorem{example}{Example}
\newtheorem{proposition}{Proposition}
\newtheorem{theorem}{Theorem}
%\newtheorem{theoremCheck}[theorem]{Theorem${}^{\text{\color{darkgreen}{\Checkmark}}}$}
\newtheorem{corollary}{Corollary}
\newtheorem{conjecture}{Conjecture}


\newcommand{\push}[1]{{\ensuremath{\textnormal{\textsf{push}}}(#1)}}
\newcommand{\pushl}[1]{{\ensuremath{\textnormal{\textsf{push}}_{\textnormal{\textsf{l}}}}(#1)}}
\newcommand{\pushr}[1]{{\ensuremath{\textnormal{\textsf{push}}_{\textnormal{\textsf{r}}}}(#1)}}
\newcommand{\pushlr}{{\ensuremath{\textnormal{\textsf{push}}_{\textnormal{\textsf{lr}}}}}}
\newcommand{\smashinc}[2]{{\ensuremath{\langle #1,#2 \rangle}}}
\newcommand{\smashincalt}[3]{{\ensuremath{\langle #1,#2,#3 \rangle}}}

\newcommand{\incl}[1]{\textnormal{\ensuremath{\mathsf{incl}_{#1}}}}
\newcommand{\fin}[1]{\textnormal{\ensuremath{\mathsf{Fin}({#1})}}}


%% Configure appearance of system names here
\newcommand{\teletype}[1]{\ensuremath{\mathtt{#1}}}
\newcommand{\systemname}[1]{\teletype{\color{darkgray}#1}\xspace}

%% Systems referred to in the paper
\newcommand{\cubicaltt}{\systemname{cubicaltt}}
\newcommand{\cooltt}{\systemname{cooltt}}
\newcommand{\Idris}{\systemname{Idris}}
\newcommand{\Haskell}{\systemname{Haskell}}
\newcommand{\Lean}{\systemname{Lean}}

\definecolor{Revolutionary}{RGB}{232,70,68}
\newcommand{\redtt}{\textbf{\teletype{{\color{Revolutionary}red}tt}}}

\newcommand{\eg}{{e.g.}}
\newcommand{\ie}{{i.e.}}
\newcommand{\cf}{{cf.}}

%% Type theory macros
\newcommand{\su}[2]{#1/#2}
\newcommand{\subst}[2]{(\su #1 #2)}
\newcommand{\esubst}[3]{(#1, \su #2 #3)}
\newcommand{\substnop}[2]{{#2}\, / \,{#1}}

%% \usepackage{MnSymbol}
\usepackage{newunicodechar}
\newunicodechar{≃}{\ensuremath{\simeq}}
\newunicodechar{≅}{\ensuremath{\mathsfg}}
\newunicodechar{∈}{\ensuremath{\in}}
\newunicodechar{⁻}{\ensuremath{^{-}}}
\newunicodechar{ᴹ}{\ensuremath{_{M}}}
\newunicodechar{ᴺ}{\ensuremath{_{N}}}
\newunicodechar{≡}{\ensuremath{\equiv}}
\newunicodechar{λ}{\ensuremath{\lambda}}
\newunicodechar{⊎}{\ensuremath{\uplus}}
\newunicodechar{∷}{\ensuremath{::}}
\newunicodechar{ℓ}{\ensuremath{\ell}}
\newunicodechar{ᵢ}{\ensuremath{_i}}
\newunicodechar{⟨}{\ensuremath{\langle}}
\newunicodechar{⟩}{\ensuremath{\rangle}}
\newunicodechar{α}{\ensuremath{\alpha}}
\newunicodechar{β}{\ensuremath{\beta}}
\newunicodechar{θ}{\ensuremath{\theta}}
\newunicodechar{φ}{\ensuremath{\varphi}}
\newunicodechar{ψ}{\ensuremath{\psi}}
\newunicodechar{η}{\ensuremath{\eta}}
\newunicodechar{ε}{\ensuremath{\varepsilon}}
\newunicodechar{ι}{\ensuremath{\iota}}
\newunicodechar{Σ}{\ensuremath{\Sigma}}
\newunicodechar{σ}{\ensuremath{\sigma}}
\newunicodechar{∀}{\ensuremath{\forall}}
\newunicodechar{ℕ}{\ensuremath{\mathbb{N}}}
\newunicodechar{→}{\ensuremath{\to}}
\newunicodechar{⊎}{\ensuremath{\uplus}}
\newunicodechar{⋆}{\ensuremath{*}}
\newunicodechar{¬}{\ensuremath{\lnot}}
\newunicodechar{Θ}{\ensuremath{\theta}}
\newunicodechar{ᴸ}{\ensuremath{{}^{\color{black}\leftarrow}}}  % hax
\newunicodechar{ᴿ}{\ensuremath{{}^{\color{black}\rightarrow}}} % hax
\newunicodechar{∧}{\ensuremath{\wedge}}
\newunicodechar{∨}{\ensuremath{\vee}}
\newunicodechar{∼}{\ensuremath{\sim}}
\newunicodechar{≢}{\ensuremath{\nequiv}}
\newunicodechar{Π}{\ensuremath{\Pi}}
\newunicodechar{ℤ}{\ensuremath{\mathbb{Z}}}
\newunicodechar{∥}{\ensuremath{\parallel}}
\newunicodechar{∣}{\ensuremath{\mid}}
\newunicodechar{ℚ}{\ensuremath{\mathbb{Q}}}
\newunicodechar{₊}{\ensuremath{_+}}
\newunicodechar{∗}{\ensuremath{\ast}}
\newunicodechar{₁}{\ensuremath{_1}}
\newunicodechar{₂}{\ensuremath{_2}}
\newunicodechar{⊥}{\ensuremath{\bot}}
\newunicodechar{∘}{\ensuremath{\circ}}
\newunicodechar{ρ}{\ensuremath{\rho}}
\newunicodechar{↦}{\ensuremath{\mapsto}}
\newunicodechar{₀}{\ensuremath{_0}}
\newunicodechar{∙}{\ensuremath{\boldsymbol{\cdot}}}
\newunicodechar{Ω}{\ensuremath{\Omega}}
\newunicodechar{§}{\ensuremath{\mathbb{S}}}
\newunicodechar{𝟙}{\ensuremath{\mathbb{1}}}
\newunicodechar{×}{\ensuremath{\times}}

%Ugly hacks for torus, klein and RP2
\newunicodechar{□}{\mathsf{\ensuremath{\square}}}
\newunicodechar{α}{\textsf{\ensuremath{{\mathbb{T}^2}}}} %Torus

\newunicodechar{β}{\textsf{\ensuremath{{\mathbb{K}^2}}}} %Klein

\newunicodechar{γ}{\textsf{\ensuremath{{\mathbb{R}P^2}}}} %RP^2


% Hacks for ex computation
\newunicodechar{ₕ}{\textsf{\ensuremath{_h}}}
\newunicodechar{ϕ}{\textsf{\ensuremath{\phi}}}
\newunicodechar{ζ}{\anum{3}\;\;\;\;\;\;\;\;\;}

\newcommand{\winding}[1]{\textsf{winding}({#1})}
\newcommand{\Type}{\textsf{Type}}
\newcommand{\ptrunc}[1]{\textsf{∥}\,{#1}\,\textsf{∥}}
\newcommand{\tyProduct}[2]{{#1}\,\AgdaOperator{\AgdaTextsftion{×}}\,{#2}}
\newcommand{\tySigma}[3]{\textsf{Σ[}\,{#1}\,\textsf{∈}\,{#2}\,\textsf{]}\,{#3}}
\newcommand{\tyMaybe}[1]{\data{Maybe}\,{#1}}
\newcommand{\tyNat}{\data{ℕ}}
\newcommand{\tyPath}[2]{{#1}\,\textsf{≡}\,{#2}}
\newcommand{\tyPathP}[4]{\primty{PathP}\,(\symb{λ}\,{#1} \to {#2})\,{#3}\,{#4}}
\newcommand{\tyEquiv}[2]{{#1}\,\textsf{≃}\,{#2}}
\newcommand{\tySum}[2]{{#1}\,\textsf{⊎}\,{#2}}
\newcommand{\tyNot}[1]{\textsf{¬}\,{#1}}
\newcommand{\tyStructEq}[3]{{#1}\,\textsf{≃[}\,{#2}\,\textsf{]}\,{#3}}



\newlength{\LETTERheight}
\AtBeginDocument{\settoheight{\LETTERheight}{I}}
\newcommand*{\longleadsto}[1]{\ \raisebox{0.24\LETTERheight}{\tikz \draw [->,
line join=round,
decorate, decoration={
    zigzag,
    segment length=4,
    amplitude=.9,
    post=lineto,
    post length=2pt
}] (0,0) -- (#1,0);}\ }

\newcommand{\cmark}{\ding{51}}%
\newcommand{\xmark}{\ding{55}}%
\newcommand{\equivDef}{\overset{\scriptscriptstyle \textsf{def}}{\equiv}}
\newcommand{\trivGrp}{\textsf{𝟙}}
\newcommand{\bZ}{\textsf{ℤ}}
\newcommand{\bN}{\textsf{$\mathbb{N}$}}
\newcommand{\sphere}[1]{\textsf{§}^{#1}}
\newcommand{\klein}{\textsf{$\mathbb{K}^{\,2}$}}
\newcommand{\cohom}[2]{H^{#1}\!\left({#2}\right)}
\newcommand{\funspace}[2]{\left({{#2} \to \textsf{K}_{#1}}\right)}
\newcommand{\funspaceP}[2]{{{#2} \to \textsf{K}_{#1}}}
\newcommand{\torus}{\textsf{$\mathbb{T}^{\,2}$}}
\newcommand{\RP}{\textsf{$\mathbb{R}P^2$}}
\newcommand{\RPinf}{\textnormal{\ensuremath{\mathbb{R}P^\infty}}}
\newcommand{\CP}{\textsf{$\mathbb{C}P^2$}}
\newcommand{\refl}{\textsf{refl}}
\newcommand{\inr}[1]{\mathsf{inr}\,{#1}}
\newcommand{\inl}[1]{\mathsf{inl}\,{#1}}
\newcommand{\trunc}[1]{\mathsf{∣}\,#1\,\mathsf{∣}}
\newcommand{\truncT}[2]{\textsf{∥}\,#2\,\textsf{∥}_{#1}}
\newcommand{\Loop}{\mathsf{loop}}
\newcommand{\base}{\mathsf{base}}
\newcommand{\north}{\mathsf{north}}
\newcommand{\south}{\mathsf{south}}
\newcommand{\Code}{\textsf{Code}}
\newcommand{\merid}[1]{\mathsf{merid}\;{#1}}
\newcommand{\transport}[2]{\textsf{transport}\;(#1)\;#2}
\newcommand{\ap}[2]{\textnormal{\textsf{ap}}_{#1}{(#2)}}
\newcommand{\app}[3]{\textnormal{\textsf{ap}$^2$}_{#1}\;{#2}\;{#3}}
\newcommand{\tmPair}[2]{{#1}\,\mathsf{,}\,{#2}}
\newcommand{\tmTrunc}[1]{\mathsf{[}\,{#1}\,\mathsf{]}}
\newcommand{\tmNil}{\mathsf{[]}}
\newcommand{\tmCons}[2]{{#1}\;\mathsf{∷}\;{#2}}

\newcommand{\cupprodop}{\textsf{$\smile$}}
\newcommand{\cupprodkop}{\textsf{$\smile_k$}}
\newcommand{\cupprodk}[2]{{#1}\;\cupprodkop\;{#2}}
\newcommand{\cupprod}[2]{{#1}\;\cupprodop\;{#2}}

%% Comment macro colors
\definecolor{dkblue}{rgb}{0,0.1,0.5}
\definecolor{lightblue}{rgb}{0,0.5,0.5}
\definecolor{dkgreen}{rgb}{0,0.6,0}
\definecolor{dkbrown}{rgb}{0.4,0,0}
\definecolor{dkviolet}{rgb}{0.6,0,0.8}

%% SUBMISSION: turn off all comments
%\newcommand{\mycomment}[3]{}
\newcommand{\mycomment}[3]{\textcolor{#1}{[#2#3]}}
\newcommand{\todo}[1]{\mycomment{red}{TODO: }{#1}}

\newcommand{\sq}[1]{\textnormal{\ensuremath{\mathsf{Sq}^{#1}}}}
\newcommand{\sqind}[2]{\textnormal{\ensuremath{\mathsf{Sq}_{#2}^{#1}}}}

\DeclareMathOperator*{\bigast}{\raisebox{-0.6ex}{\scalebox{2.5}{$\ast$}}}
\DeclareMathOperator*{\bigsmile}{\raisebox{-0.ex}{\scalebox{1.5}{$\smile$}}}

\newcommand{\negpath}{\textnormal{\textsf{neg-path}}}
\newcommand{\totSq}{\textnormal{\ensuremath{\widehat{\mathsf{Sq}}}}}

\newcommand{\cwskel}{\textnormal{\ensuremath{\mathsf{CW}^{\mathsf{skel}}}}}
\newcommand{\cwskelinf}{\textnormal{\ensuremath{\mathsf{CW}_\infty^{\mathsf{skel}}}}}
\newcommand{\hskel}[1]{\textnormal{\ensuremath{H^{\mathsf{skel}}_{#1}}}}
\newcommand{\hskelinf}[1]{\textnormal{\ensuremath{H^{\mathsf{skel}_\infty}_{#1}}}}
\newcommand{\abgrp}{\textnormal{\ensuremath{\mathsf{AbGrp}}}}


\title{Cellular Homology and the Cellular Approximation Theorem}

\author{
  Axel Ljungström\\
  \footnotesize{Stockholm University, Sweden}\\
  \footnotesize{\texttt{axel.ljungstrom@math.su.se}}
  \and
  Anders Mörtberg\\
  \footnotesize{Stockholm University, Sweden}\\
  \footnotesize{\texttt{anders.mortberg@math.su.se}}
  \and
  Loïc Pujet\\
  \footnotesize{Stockholm University, Sweden}\\
  \footnotesize{\texttt{loic@pujet.fr}}
}
\date{}

\begin{document}

\maketitle
\vspace{-.5cm}
%
In~\cite{BuchholtzFavonia18}, Buchholtz and Favonia develop a theory
of cellular cohomology in HoTT. The authors proceed in two steps:
%
first, they define the cohomology groups of a CW complex, employing the
standard definition in terms of cochain complexes (see e.g.\ \cite{May1999}),
and then, they construct of an isomorphism between their definition and
the, in HoTT, more well-established cohomology theory defined in terms of Eilenberg-MacLane
spaces \cite{ShulmanBlog13,LicataFinster14,CavalloMsc15}.
%
This second step allows the authors to derive many properties of their
cohomology theory (e.g.\ functoriality and the Eilenberg-Steenrod axioms)
simply by transporting the relevant proofs from Eilenberg-MacLane cohomology.
%
However, this strategy is not as readily available when developing cellular \emph{homology}:
%
even though we can define homology theories in terms of Eilenberg-MacLane spaces in HoTT~\cite{graham18,christensen2020hurewicz,FlorisPhd,spectralsequences}, this is significantly more involved than for cohomology. This suggests that, perhaps, a direct construction of cellular homology is the more feasible alternative.

In this work, we revisit Buchholtz and Favonia's
definition of cellular chain complexes, from which we define a functorial homology theory. This is done not via reduction to another more
well-studied definition, but by developing the theory of CW complexes and
cellular maps.
%
In particular, we prove a constructive version of the \emph{cellular
  approximation theorem}, a cornerstone of the classical theory of CW complexes. All results presented here have been formalised in \CubicalAgda \cite{cubicalagda2}.
\todo{Add a ref to the code?}

\smallskip

In what follows, let $\mathsf{Fin}(n)$ denote the finite
set with $n$ elements. We will need the following definition in order to define CW complexes:\footnote{This definition is slightly different from the
  recursive definition employed in (the formalisation
  of)~\cite{BuchholtzFavonia18}. Its usefulness is two-fold: first, it
  allows us to also define infinite dimensional CW complexes, such as
  $\RPinf$. Second, it allows us to extract the $n$-skeleton,
  $X_n$, of $X_{\bullet}$ directly without having to rely on
  auxilliary functions. A similar reformulation can be found in \url{https://github.com/CMU-HoTT/serre-finiteness/blob/cellular/Cellular/CellComplex.agda}.}
  %
%
\begin{figure}[H]
\begin{minipage}{0.65 \linewidth}
\begin{definition}[CW skeleta]
  A \textbf{CW-skeleton} is an infinite sequence of types and maps
  $(X_{-1} \xrightarrow{\incl{-1}} X_0 \xrightarrow{\incl{0}} X_1 \xrightarrow{\incl{1}} \dots )$
  equipped with a function $c : \bN\cup\{-1\} \to \bN$ and a set of attaching maps $\alpha_i : S^i \times \fin{c_i} \to X_i$ s.t.\ $X_{-1}$ is empty and the square on the right is a (homotopy) pushout. A CW-skeleton is said to be \textbf{finite (of dimension $n$)} if $\incl{m}$ is
an equivalence for all $m \geq n$.
\end{definition}
\end{minipage}
\begin{minipage}{0.35 \linewidth}
  %
  \[
\begin{tikzcd}[ampersand replacement=\& , row sep = small]
	{S^{i} \times \fin{c_i}} \&\& \fin{c_i} \\
	{X_i} \&\& {X_{i+1}}
	\arrow["{\mathsf{snd}}", from=1-1, to=1-3]
	\arrow["{\alpha_{i}}"', from=1-1, to=2-1]
	\arrow[from=2-1, to=2-3]
	\arrow[from=1-3, to=2-3]
	\arrow["\lrcorner"{anchor=center, pos=0.125, rotate=180}, draw=none, from=2-3, to=1-1]
\end{tikzcd}
\]
\end{minipage}
\end{figure}
%
The pushout condition ensures that the \( (i+1) \)-skeleton \( X_{i+1} \) is
obtained by attaching a finite number of \( i \)-dimensional cells to the
\( i \)-skeleton \( X_i \).
%
We will often simply write $X_\bullet$ for a CW skeleton $(X_0,X_1,\dots)$ and
take $\incl{\bullet}$, $c_{\bullet}$ and $ \alpha_{\bullet}$ to be implicit.
%
We denote by $\cwskelinf$ the wild category whose objects are CW skeleta and
whose morphisms are maps between their sequential colimits, i.e.\
$\mathsf{Hom}(X_\bullet,Y_\bullet) := (X_\infty \to Y_\infty)$.
%
We denote by $\cwskel$ the wild category with the same objects but whose
morphisms are \emph{cellular maps}:
%
\begin{figure}[H]
\vspace{-.2cm}
\begin{minipage}[t]{0.7 \linewidth}
\begin{definition}
  Let $X_\bullet$ and $Y_\bullet$ be CW skeleta. A \textbf{cellular map}, which
  we denote by $f_\bullet : X_\bullet \to Y_\bullet$, consists of a family of
  functions $f_i : X_i \to Y_i$ for each $i \geq -1$ along with a family of
  homotopies \( h_i \) making the diagram on the right commute.
\end{definition}
\end{minipage}
\hspace{.15cm}
\begin{minipage}[t]{0.3 \linewidth}
  \vspace{-.65cm}
  \[
\begin{tikzcd}[ampersand replacement=\& , row sep = small]
	{X_{i+1}} \&\& {Y_{i+1}} \\
	{X_i} \&\& {Y_i}
	\arrow["{f_{i+1}}", from=1-1, to=1-3]
	\arrow["{f_{i}}"', from=2-1, to=2-3]
	\arrow[equal, "{h_{i}}"', from=1-1, to=2-3, shorten=3ex]
	\arrow[hook, from=2-1, to=1-1]
	\arrow[hook, from=2-3, to=1-3]
\end{tikzcd}
\]
\end{minipage}
\end{figure}
\begin{definition}[CW complexes]
  A type $A$ is said to be a \textbf{CW complex} if there merely exists some
  CW skeleton $X_\bullet$ s.t.\ $A$ is equivalent to the sequential colimit
  of $X_\bullet$, i.e.\ $A \simeq X_\infty$.
  %% We say that $A$ is of \textbf{dimension $n$} if $X_\bullet$ is finite of
  %% dimension $n$ and that $A$ is \textbf{finite} if there merely exists some
  %% $n$ such that $A$ is of dimension $n$.
\end{definition}

Let $\bZ[n]$ denote the free abelian group with $n-1$ generators, with
$\bZ[0]$ defined to be the trivial group. Buchholtz and
Favonia~\cite{BuchholtzFavonia18} showed how to construct the chain
complex associated to a CW-skeleton:
$
\dots \xrightarrow{\partial_3} \bZ[c_{2}]
\xrightarrow{\partial_2} \bZ[c_{1}]
\xrightarrow{\partial_1} \bZ[c_{0}]
\xrightarrow{\partial_0} 0
$.
%
We can show that \( \partial_n \circ \partial_{n+1} = 0 \) for all \( n \),
which allows us to define the \( n \)-th homology group of a CW-skeleton by
$\hskel{n}(X) := \mathsf{ker}(\partial_n)/\mathsf{im}(\partial_{n+1})$. The differentials $\partial_n$ are defined in terms of (an appropriate definition of) the \emph{degree} of map between wedge sums of spheres $\bigvee_{x : \mathsf{Fin}(c_{n+1})} \mathbb{S}^n \to \bigvee_{x : \mathsf{Fin}(c_{n})} \mathbb{S}^n$. In fact, much of our work can be reduced to statements about the behaviour of such functions and of the degree assignment.

The homology groups defined here can be extended to functors from \( \cwskel \)
to \( \abgrp \):
\todo{Maybe be more explicit about what this means as $\cwskel$ is wild?}
%
\begin{proposition}
  $\hskel{n}$ is functorial.
\end{proposition}
The argument is standard. We can transform a cellular map into an intertwining map
  between chain complexes, from which we get a homomorphism of homology groups.

Now, in order to get a homology theory on CW complexes, we would like to lift
this homology functor from $\cwskel$ to $\cwskelinf$.
%
We can straightforwardly define $\hskelinf{n} : \cwskelinf \to \abgrp$ on
objects by $\hskelinf{n}(X) := H_n^{\mathsf{skel}}(X)$. The action on
morphisms, however, is less obvious: in order to reuse the functoriality of $\hskel{n}$,
we need a way to lift maps $X_\infty \to Y_\infty$ to cellular maps
$X_\bullet \to Y_\bullet$.
%
In the classical theory of CW complexes, this is the role of the \emph{cellular
approximation theorem} \cite[Section 10.4]{May1999}.
%
However, this theorem critically relies on the axiom of choice, so we cannot prove it as is if we want to be constructive.
%
Fortunately, we are still allowed to use finite choice, which lets us prove a
version which is restricted to the case where $X_\bullet$ and $Y_\bullet$ are
finite:
%
%% Given a CW skeleton $X_\bullet$, let $(X_i^{(n)})$ denote the subcomplex obtained by setting $X_i^{(n)} = \begin{cases} X_i & \text{ if $x \leq n$} \\ X_n &\text{ otherwise} \end{cases}$.
\begin{theorem}[Cellular approximation, part 1]\label{cell-approx}
  Let $X_\bullet$ and $Y_\bullet$ be two finite CW skeleta and
  $f : X_{\infty} \to Y_{\infty}$ a map between their colimits.
  %
  There merely exists a cellular map $f_\bullet : X_\bullet \to Y_\bullet$ s.t.\ $f_\infty = f$.
\end{theorem}
%
By \cref{cell-approx}, it suffices to define the functorial action of
$\hskelinf{n}$ on functions (of finite complexes) $f : X_\infty \to Y_\infty$
s.t.\ $f$ is merely equal to $f_\infty$ for some cellular map
$f_i : X_\bullet \to Y_\bullet$.
%
By the rule of set-valued elimination of propositional truncations, it
suffices to define $\hskelinf{n}(f_\infty)$ for $f_i : X_\bullet \to Y_\bullet$
and prove that, if $f_\infty = g_\infty$, then
$\hskelinf{n}(f_\infty) = \hskelinf{n}(g_\infty)$.
%
We define $\hskelinf{n}(f_\infty) := \hskel{n}(f_\bullet)$. To complete the
definition, we need to show that, if $f_\infty = g_\infty$, then
$\hskel{n}(f_\bullet) = \hskel{n}(g_\bullet)$. To this end, we need
to extend the approximation theorem to \emph{cellular homotopies}:
%
\begin{figure}[H]
  \begin{minipage}{0.41 \linewidth}
\begin{definition}
  A \textbf{cellular homotopy} between two cellular maps
  $f_\bullet, g_\bullet : X_\bullet \to Y_\bullet$, denoted $f_\bullet \sim g_\bullet$, is a
  family of paths $h_i : (x : X_i) \to \incl{i}(f_i(x)) =_{Y_{i+1}} \incl{i}(g_i(x))$
  together with fillers, for each $x:X_i$, of the square indicated on the right.
\end{definition}
  \end{minipage}
\hspace{.15cm}
\begin{minipage}{0.5 \linewidth}
  \begin{tikzcd}[ampersand replacement=\&, row sep = small]
	{\incl{i+1}(f_{i+1}(\incl{i}(x)))} \&\& {\incl{i+1}(f_{i+1}(\incl{i}(x)))} \\
	\\
	{\incl{i+1}(\incl{i}(f_{i}(x)))} \&\& {\incl{i+1}(\incl{i}(g_{i}(x)))}
	\arrow["{h_{i+1}(\incl{i}(x))}", from=1-1, to=1-3]
	\arrow["{\mathsf{ap}_{\incl{}}(h_i(x))}"', from=3-1, to=3-3]
	\arrow[from=3-3, to=1-3]
	\arrow[from=3-1, to=1-1]
  \end{tikzcd}
\end{minipage}
\end{figure}
\begin{proposition}\label{cellhomToHomology}
  If $\truncT{}{f_\bullet \sim g_\bullet}$, then $\hskel{n}(f_{\bullet}) = \hskel{n}(g_{\bullet})$.
\end{proposition}
This follows from a technical, but standard, argument. The final component is:  % a cellular approximation theorem for cellular homotopies.
\begin{theorem}[Cellular approximation, part 2]\label{cellhom-approx}
  Let $X_\bullet$ and $Y_\bullet$ be finite CW skeleta with two cellular maps $f_\bullet, g_\bullet : X_\bullet \to Y_\bullet$ s.t.\ $f_\infty = g_\infty$. In this case, there merely exists a cellular homotopy $f_\bullet \sim g_\bullet$.
\end{theorem}
Combining~\cref{cellhom-approx} and \cref{cellhomToHomology}, we see that if $f_\infty = g_\infty$, then $\hskel{n}(f_\bullet) = \hskel{n}(g_\bullet)$, which completes the definition of the functorial action of $\hskelinf{n}$ on maps between finite complexes. In order to extend this to maps between possibly infinite complexes, one simply notes that $\hskelinf{n}(X_\bullet) \cong \hskelinf{n}(X_\bullet^{(n + 1)})$ where $X_{\bullet}^{(m)}$ denotes the finite subcomplex of $X_{i}$, converging at level $m$.

Thus, we have defined the functor $\hskelinf{n}$, assigning homology groups to any type equivalent to the colimit of a CW skeleton. However, for CW complexes, the existence of such an equivalence is only assumed to \emph{merely} exist. We would like to define a fuctor $\hcw{n} : \mathsf{CW} \to \abgrp$, but the universe of abelian groups is a groupoid. We may, however, apply the rule for groupoid-valued elimination of propositional truncations. Applied to the goal in question, it says that we may define $\hcw{n}$ by (1) defining $\hcw{n}(X_\infty)$ for CW-skelta $X_\bullet$, (2) showing that for $e : X_\infty \simeq Y_{\infty}$, we have an isomorphism $e_* : \hcw{n}(X_\infty) \simeq \hcw{n}(Y_\infty)$ and (3) that $e_*$ is functorial. For (1), we simply set $\hcw{n}(X_\infty) := \hskelinf{n}(X_\bullet)$. The conditions (2) and (3) now following from functoriality of $\hskelinf{n}$. Functoriality of $\hcw{n}$ follows in a similar manner. This completes the definition of the cellular cohomology \todo{homology??} theory $\hcw{n}$.

The formalisation of the analoguous cohomology theory as well as the verification of the Eilenberg-Steenrod axioms is future/ongoing work. When this is completed, we are planning on computing cellular (co)homology groups of some well-known spaces. Using this, we are planning on putting our theory to the test in \texttt{Cubical Agda} by attepting to carry out concrete (co)homology computations. Our hope is that the development of cellular (co)homology will perform better than other alternatives and will be able to compute e.g.\ some of the examples that failed in~\cite[Section 6]{BLM22}.
\todo{Also mention in one sentence that this is part of a bigger project to formalize the proof of Barton-Campion?}

\printbibliography


\end{document}
