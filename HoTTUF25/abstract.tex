\documentclass[a4page]{article}
% \usepackage{biblatex}
\usepackage[a4paper,
            bindingoffset=0.2in,
            left=1in,
            right=1in,
            top=1in,
            bottom=1in,
            footskip=.25in]{geometry}
\usepackage{amsmath, amsfonts, amsthm , mathtools , bbold, float}

\usepackage{tikz}
\usepackage{tikz-cd}
\usepackage{mathpartir}
\usepackage[citecolor=blue,linkcolor=blue,urlcolor=blue,colorlinks]{hyperref}
\usepackage[nameinlink,capitalize,noabbrev]{cleveref}
\usepackage{quiver}
\usepackage{xspace}
\usepackage{xcolor}
\usepackage{bbm}
\usepackage{breakurl}
\usepackage{bbding}
\hypersetup{pdfborder={0 0 100}}
\usepackage{lipsum}
%\usepackage{biblatex}
%\bibliographystyle{unsrt}
\usepackage[backend=biber,style=alphabetic,sorting=nty,giveninits]{biblatex} % Bibliography management
\addbibresource{refs.bib}

\usepackage{textcomp}
\usepackage{xspace}
\usepackage{url}
\usepackage{multirow}
\usepackage{hhline}
\usepackage{pifont}
\usetikzlibrary{arrows,matrix,decorations.pathmorphing,
  decorations.markings, calc, backgrounds}
\usepackage{mathpartir}
\usepackage{microtype}

\DisableLigatures[-]{family=tt*}

\definecolor{darkgreen}{rgb}{.30,.70,.60}

\newtheorem{remark}{Remark}
\newtheorem{lemma}{Lemma}
\newtheorem{definition}{Definition}
\newtheorem{example}{Example}
\newtheorem{proposition}{Proposition}
\newtheorem{theorem}{Theorem}
%\newtheorem{theoremCheck}[theorem]{Theorem${}^{\text{\color{darkgreen}{\Checkmark}}}$}
\newtheorem{corollary}{Corollary}
\newtheorem{conjecture}{Conjecture}


\newcommand{\push}[1]{{\ensuremath{\textnormal{\textsf{push}}}(#1)}}
\newcommand{\pushl}[1]{{\ensuremath{\textnormal{\textsf{push}}_{\textnormal{\textsf{l}}}}(#1)}}
\newcommand{\pushr}[1]{{\ensuremath{\textnormal{\textsf{push}}_{\textnormal{\textsf{r}}}}(#1)}}
\newcommand{\pushlr}{{\ensuremath{\textnormal{\textsf{push}}_{\textnormal{\textsf{lr}}}}}}
\newcommand{\smashinc}[2]{{\ensuremath{\langle #1,#2 \rangle}}}
\newcommand{\smashincalt}[3]{{\ensuremath{\langle #1,#2,#3 \rangle}}}

\newcommand{\incl}[1]{\textnormal{\ensuremath{\mathsf{incl}_{#1}}}}
\newcommand{\fin}[1]{\textnormal{\ensuremath{\mathsf{Fin}({#1})}}}


%% Configure appearance of system names here
\newcommand{\teletype}[1]{\ensuremath{\mathtt{#1}}}
\newcommand{\systemname}[1]{\teletype{\color{darkgray}#1}\xspace}

%% Systems referred to in the paper
\newcommand{\cubicaltt}{\systemname{cubicaltt}}
\newcommand{\cooltt}{\systemname{cooltt}}
\newcommand{\Idris}{\systemname{Idris}}
\newcommand{\Haskell}{\systemname{Haskell}}
\newcommand{\Lean}{\systemname{Lean}}

\definecolor{Revolutionary}{RGB}{232,70,68}
\newcommand{\redtt}{\textbf{\teletype{{\color{Revolutionary}red}tt}}}

\newcommand{\eg}{{e.g.}}
\newcommand{\ie}{{i.e.}}
\newcommand{\cf}{{cf.}}

%% Type theory macros
\newcommand{\su}[2]{#1/#2}
\newcommand{\subst}[2]{(\su #1 #2)}
\newcommand{\esubst}[3]{(#1, \su #2 #3)}
\newcommand{\substnop}[2]{{#2}\, / \,{#1}}

%% \usepackage{MnSymbol}
\usepackage{newunicodechar}
\newunicodechar{≃}{\ensuremath{\simeq}}
\newunicodechar{≅}{\ensuremath{\mathsfg}}
\newunicodechar{∈}{\ensuremath{\in}}
\newunicodechar{⁻}{\ensuremath{^{-}}}
\newunicodechar{ᴹ}{\ensuremath{_{M}}}
\newunicodechar{ᴺ}{\ensuremath{_{N}}}
\newunicodechar{≡}{\ensuremath{\equiv}}
\newunicodechar{λ}{\ensuremath{\lambda}}
\newunicodechar{⊎}{\ensuremath{\uplus}}
\newunicodechar{∷}{\ensuremath{::}}
\newunicodechar{ℓ}{\ensuremath{\ell}}
\newunicodechar{ᵢ}{\ensuremath{_i}}
\newunicodechar{⟨}{\ensuremath{\langle}}
\newunicodechar{⟩}{\ensuremath{\rangle}}
\newunicodechar{α}{\ensuremath{\alpha}}
\newunicodechar{β}{\ensuremath{\beta}}
\newunicodechar{θ}{\ensuremath{\theta}}
\newunicodechar{φ}{\ensuremath{\varphi}}
\newunicodechar{ψ}{\ensuremath{\psi}}
\newunicodechar{η}{\ensuremath{\eta}}
\newunicodechar{ε}{\ensuremath{\varepsilon}}
\newunicodechar{ι}{\ensuremath{\iota}}
\newunicodechar{Σ}{\ensuremath{\Sigma}}
\newunicodechar{σ}{\ensuremath{\sigma}}
\newunicodechar{∀}{\ensuremath{\forall}}
\newunicodechar{ℕ}{\ensuremath{\mathbb{N}}}
\newunicodechar{→}{\ensuremath{\to}}
\newunicodechar{⊎}{\ensuremath{\uplus}}
\newunicodechar{⋆}{\ensuremath{*}}
\newunicodechar{¬}{\ensuremath{\lnot}}
\newunicodechar{Θ}{\ensuremath{\theta}}
\newunicodechar{ᴸ}{\ensuremath{{}^{\color{black}\leftarrow}}}  % hax
\newunicodechar{ᴿ}{\ensuremath{{}^{\color{black}\rightarrow}}} % hax
\newunicodechar{∧}{\ensuremath{\wedge}}
\newunicodechar{∨}{\ensuremath{\vee}}
\newunicodechar{∼}{\ensuremath{\sim}}
\newunicodechar{≢}{\ensuremath{\nequiv}}
\newunicodechar{Π}{\ensuremath{\Pi}}
\newunicodechar{ℤ}{\ensuremath{\mathbb{Z}}}
\newunicodechar{∥}{\ensuremath{\parallel}}
\newunicodechar{∣}{\ensuremath{\mid}}
\newunicodechar{ℚ}{\ensuremath{\mathbb{Q}}}
\newunicodechar{₊}{\ensuremath{_+}}
\newunicodechar{∗}{\ensuremath{\ast}}
\newunicodechar{₁}{\ensuremath{_1}}
\newunicodechar{₂}{\ensuremath{_2}}
\newunicodechar{⊥}{\ensuremath{\bot}}
\newunicodechar{∘}{\ensuremath{\circ}}
\newunicodechar{ρ}{\ensuremath{\rho}}
\newunicodechar{↦}{\ensuremath{\mapsto}}
\newunicodechar{₀}{\ensuremath{_0}}
\newunicodechar{∙}{\ensuremath{\boldsymbol{\cdot}}}
\newunicodechar{Ω}{\ensuremath{\Omega}}
\newunicodechar{§}{\ensuremath{\mathbb{S}}}
\newunicodechar{𝟙}{\ensuremath{\mathbb{1}}}
\newunicodechar{×}{\ensuremath{\times}}

%Ugly hacks for torus, klein and RP2
\newunicodechar{□}{\mathsf{\ensuremath{\square}}}
\newunicodechar{α}{\textsf{\ensuremath{{\mathbb{T}^2}}}} %Torus

\newunicodechar{β}{\textsf{\ensuremath{{\mathbb{K}^2}}}} %Klein

\newunicodechar{γ}{\textsf{\ensuremath{{\mathbb{R}P^2}}}} %RP^2


% Hacks for ex computation
\newunicodechar{ₕ}{\textsf{\ensuremath{_h}}}
\newunicodechar{ϕ}{\textsf{\ensuremath{\phi}}}
\newunicodechar{ζ}{\anum{3}\;\;\;\;\;\;\;\;\;}

\newcommand{\winding}[1]{\textsf{winding}({#1})}
\newcommand{\Type}{\textsf{Type}}
\newcommand{\ptrunc}[1]{\textsf{∥}\,{#1}\,\textsf{∥}}
\newcommand{\tyProduct}[2]{{#1}\,\AgdaOperator{\AgdaTextsftion{×}}\,{#2}}
\newcommand{\tySigma}[3]{\textsf{Σ[}\,{#1}\,\textsf{∈}\,{#2}\,\textsf{]}\,{#3}}
\newcommand{\tyMaybe}[1]{\data{Maybe}\,{#1}}
\newcommand{\tyNat}{\data{ℕ}}
\newcommand{\tyPath}[2]{{#1}\,\textsf{≡}\,{#2}}
\newcommand{\tyPathP}[4]{\primty{PathP}\,(\symb{λ}\,{#1} \to {#2})\,{#3}\,{#4}}
\newcommand{\tyEquiv}[2]{{#1}\,\textsf{≃}\,{#2}}
\newcommand{\tySum}[2]{{#1}\,\textsf{⊎}\,{#2}}
\newcommand{\tyNot}[1]{\textsf{¬}\,{#1}}
\newcommand{\tyStructEq}[3]{{#1}\,\textsf{≃[}\,{#2}\,\textsf{]}\,{#3}}



\newlength{\LETTERheight}
\AtBeginDocument{\settoheight{\LETTERheight}{I}}
\newcommand*{\longleadsto}[1]{\ \raisebox{0.24\LETTERheight}{\tikz \draw [->,
line join=round,
decorate, decoration={
    zigzag,
    segment length=4,
    amplitude=.9,
    post=lineto,
    post length=2pt
}] (0,0) -- (#1,0);}\ }

\newcommand{\cmark}{\ding{51}}%
\newcommand{\xmark}{\ding{55}}%
\newcommand{\equivDef}{\overset{\scriptscriptstyle \textsf{def}}{\equiv}}
\newcommand{\trivGrp}{\textsf{𝟙}}
\newcommand{\bZ}{\textsf{ℤ}}
\newcommand{\bN}{\textsf{$\mathbb{N}$}}
\newcommand{\sphere}[1]{\textsf{§}^{#1}}
\newcommand{\klein}{\textsf{$\mathbb{K}^{\,2}$}}
\newcommand{\cohom}[2]{H^{#1}\!\left({#2}\right)}
\newcommand{\funspace}[2]{\left({{#2} \to \textsf{K}_{#1}}\right)}
\newcommand{\funspaceP}[2]{{{#2} \to \textsf{K}_{#1}}}
\newcommand{\torus}{\textsf{$\mathbb{T}^{\,2}$}}
\newcommand{\RP}{\textsf{$\mathbb{R}P^2$}}
\newcommand{\RPinf}{\textnormal{\ensuremath{\mathbb{R}P^\infty}}}
\newcommand{\CP}{\textsf{$\mathbb{C}P^2$}}
\newcommand{\refl}{\textsf{refl}}
\newcommand{\inr}[1]{\mathsf{inr}\,{#1}}
\newcommand{\inl}[1]{\mathsf{inl}\,{#1}}
\newcommand{\trunc}[1]{\mathsf{∣}\,#1\,\mathsf{∣}}
\newcommand{\truncT}[2]{\textsf{∥}\,#2\,\textsf{∥}_{#1}}
\newcommand{\Loop}{\mathsf{loop}}
\newcommand{\base}{\mathsf{base}}
\newcommand{\north}{\mathsf{north}}
\newcommand{\south}{\mathsf{south}}
\newcommand{\Code}{\textsf{Code}}
\newcommand{\merid}[1]{\mathsf{merid}\;{#1}}
\newcommand{\transport}[2]{\textsf{transport}\;(#1)\;#2}
\newcommand{\ap}[2]{\textnormal{\textsf{ap}}_{#1}{(#2)}}
\newcommand{\app}[3]{\textnormal{\textsf{ap}$^2$}_{#1}\;{#2}\;{#3}}
\newcommand{\tmPair}[2]{{#1}\,\mathsf{,}\,{#2}}
\newcommand{\tmTrunc}[1]{\mathsf{[}\,{#1}\,\mathsf{]}}
\newcommand{\tmNil}{\mathsf{[]}}
\newcommand{\tmCons}[2]{{#1}\;\mathsf{∷}\;{#2}}

\newcommand{\cupprodop}{\textsf{$\smile$}}
\newcommand{\cupprodkop}{\textsf{$\smile_k$}}
\newcommand{\cupprodk}[2]{{#1}\;\cupprodkop\;{#2}}
\newcommand{\cupprod}[2]{{#1}\;\cupprodop\;{#2}}

%% Comment macro colors
\definecolor{dkblue}{rgb}{0,0.1,0.5}
\definecolor{lightblue}{rgb}{0,0.5,0.5}
\definecolor{dkgreen}{rgb}{0,0.6,0}
\definecolor{dkbrown}{rgb}{0.4,0,0}
\definecolor{dkviolet}{rgb}{0.6,0,0.8}

%% SUBMISSION: turn off all comments
%\newcommand{\mycomment}[3]{}
\newcommand{\mycomment}[3]{\textcolor{#1}{[#2#3]}}
\newcommand{\todo}[1]{\mycomment{red}{TODO: }{#1}}

\newcommand{\sq}[1]{\textnormal{\ensuremath{\mathsf{Sq}^{#1}}}}
\newcommand{\sqind}[2]{\textnormal{\ensuremath{\mathsf{Sq}_{#2}^{#1}}}}

\DeclareMathOperator*{\bigast}{\raisebox{-0.6ex}{\scalebox{2.5}{$\ast$}}}
\DeclareMathOperator*{\bigsmile}{\raisebox{-0.ex}{\scalebox{1.5}{$\smile$}}}

\newcommand{\negpath}{\textnormal{\textsf{neg-path}}}
\newcommand{\totSq}{\textnormal{\ensuremath{\widehat{\mathsf{Sq}}}}}

\newcommand{\cwskel}{\textnormal{\ensuremath{\mathsf{CW}^{\mathsf{skel}}}}}
\newcommand{\cwskelinf}{\textnormal{\ensuremath{\mathsf{CW}_\infty^{\mathsf{skel}}}}}
\newcommand{\hskel}[1]{\textnormal{\ensuremath{H^{\mathsf{skel}}_{#1}}}}
\newcommand{\hskelinf}[1]{\textnormal{\ensuremath{H^{\mathsf{skel}_\infty}_{#1}}}}
\newcommand{\abgrp}{\textnormal{\ensuremath{\mathsf{AbGrp}}}}


\title{Hurewicz and Brouwer}

\author{
  Axel Ljungström\\
  \footnotesize{Stockholm University, Sweden}\\
  \footnotesize{\texttt{axel.ljungstrom@math.su.se}}
  \and
  Loïc Pujet\\
  \footnotesize{Stockholm University, Sweden}\\
  \footnotesize{\texttt{loic@pujet.fr}}
}
\date{}

\begin{document}

\maketitle
\vspace{-.5cm}
%
Witold Hurewicz is one of the most important names in algebraic topology.
Most famously, he is credited with the definition of higher homotopy groups and with the
\emph{Hurewicz theorem} which establishes a relation between these and homology groups.
%
Interestingly, Hurewicz worked from 1927 to 1936 as an assistant of L.E.J. Brouwer,
the father of intuitionism.
%
However, it seems that Hurewicz did not inherit his mentor's interest for constructive proofs.
%
In this abstract, we will do our best to posthumously reconcile these two mathematicians
by giving a constructive proof of the Hurewicz theorem in Homotopy Type Theory.

We should note that Christensen and Scoccola have already proved a version of the Hurewicz
theorem in HoTT~\cite{christensen2020hurewicz}.
%
First, they work with a homology theory defined in terms of stable
homotopy groups, while we work with cellular homology (two theories
which are not known to be equivalent in HoTT).
%
Second, their proof boils down to connectedness properties of the
smash product, while our strategy relies on the argument that \( n
\)-connected spaces may be approximated by \emph{CW complexes} with no
nontrivial cells in dimension~\( \le n \) -- an argument which
somehwat surprisingly holds constructively and by all means is interesting in its own right.
%
We borrow the definition of CW complexes from last years
submission~\cite{HoTTUF} which in turn is based on Buchholtz and
Favonia's definition~\cite{BuchholtzFavonia18}.
\begin{figure}[H]
\begin{minipage}{0.68 \linewidth}
\begin{definition}
  A \textbf{CW structure} is a sequence of types and maps
  $(X_{-1} \xrightarrow{\incl{-1}} X_0 \xrightarrow{\incl{0}} X_1 \xrightarrow{\incl{1}} \dots )$
  equipped with a function $c : \bN \to \bN$ and a set of attaching maps $\alpha_i : S^i \times \fin{c_{i+1}} \to X_i$ for $i \geq -1$ s.t. $X_{-1}$
\end{definition}
\end{minipage}
\vspace{-.1cm}
\hspace{.1cm}
\begin{minipage}{0.32 \linewidth}
  %
\begin{tikzcd}[ampersand replacement=\& , column sep = small , row sep = small]
	{S^{i} \times \fin{c_{i+1}}} \&\& \fin{c_{i+1}} \\
	{X_i} \&\& {X_{i+1}}
	\arrow["{\mathsf{snd}}", from=1-1, to=1-3]
	\arrow["{\alpha_{i}}"', from=1-1, to=2-1]
	\arrow[from=2-1, to=2-3]
	\arrow[from=1-3, to=2-3]
	\arrow["\lrcorner"{anchor=center, pos=0.125, rotate=180}, draw=none, from=2-3, to=1-1]
\end{tikzcd}
\end{minipage}
\emph{is empty and the square on the right is a pushout.
  A type $A$ is said to be a \textbf{CW complex} if there merely exists some
  CW structure $X_\ast$ s.t.\ $A$ is equivalent to the sequential colimit
  of $X_\ast$, i.e.\ $A \simeq X_\infty$.}
\end{figure}
%
Informally, a CW complex is a type that can be built by an iterative process of attaching
cells: we start with a finite number of 0-dimensional points, then we add a finite
amount of 1-dimensional edges, then we add a finite amount of 2-dimensional discs, and so on.
%
These explicit descriptions are quite convenient for defining properties and operations by
induction on dimensions. Most notably, Buchholtz and Favonia showed how to define the
\emph{cellular (co)homology groups} of any CW complex.
%
Recently, we proved that their definition of cellular homology groups forms a homology theory.

\begin{proposition}
  Given any integer \( n \ge 0 \), one can define a functor \( \widetilde{H}_n \) from the
  category of CW complexes to the category of abelian groups.
  Moreover, the resulting family of functors satisfy the Eilenberg-Steenrod axioms for reduced homology.
\end{proposition}

Our goal here, however, is to prove the \emph{cellular Hurewicz theorem}, which informally states that if a CW complex is
\( (n - 1) \)\nobreakdash-connected, then its cellular homology groups coincide with its homotopy groups up to
dimension \( n \) (up to abelianisation in the case \( n = 1 \)).
%
The classical proof of this theorem takes an arbitrary \( (n - 1) \)-connected CW complex, and replaces
its CW structure with an alternative one with no nontrivial cells in dimension~\( < n \).
%
This is done by defining the new set of \( n \)-cells to be the \( n \)th
homotopy group of the space, from which the Hurewicz theorem will follow.
%
Unfortunately, this approach will not work with our definition of CW complexes which only allows
a finite number of cells in every dimension, since finite CW complexes tend to have infinite
homotopy groups (think of the \( n \)-sphere for instance).
%
Thankfully, it turns out that a slightly different approximation theorem is provable without leaving
the world of finite CW complexes.

\begin{definition}
  We say that a CW structure $X_\ast$ is \textbf{Hurewicz $n$-connected} if $c^X_0 = 1$ and $c^X_i = 0$ for $0<i\le n$.
  We use the same terminology for CW complexes which merely have a Hurewicz $n$-connected CW structure.
\end{definition}

\begin{theorem}\label{cor:hurewicz-con}
  A CW complex is $n$-connected if and only if it is Hurewicz $n$-connected.
\end{theorem}

\begin{proof}
A formal proof is available at \url{https://github.com/agda/cubical/blob/master/Cubical/CW/Connected.agda}.
\end{proof}

With this approximation theorem, we are in a position to compute the homology and the homotopy
of \( (n - 1) \)-connected CW complexes.
%
First, say that a type is an \( n \)-dimensional \emph{sphere bouquet} if it can be
written as \( \bouquetGen{A}{n} \) for some finite set \( A \).
%
Then, \cref{cor:hurewicz-con} implies that for any \( (n - 1) \)-connected CW complex \( X_\ast \), the
\( (n+1) \)-skeleton \( X_{n+1} \) is equivalent to the cofibre of a pointed map of
\( n \)-dimensional sphere bouquets:
%
\[
X_{n+1} \simeq \mathsf{cofib}\Big( \bouquetGen{A}{n} \xrightarrow{\ f\ }_\star \bouquetGen{B}{n} \Big).
\]
%
Now, note that \( \hredcw{n}(X_\ast) = \hredcw{n}(X_{n+1})\) for reasons of connectedness.
%
We can compute this group using the exactness property of cellular homology:
consider the sequence
%
\[
\bouquetGenL{A}{n} \xrightarrow{f} \bouquetGenL{B}{n} \xrightarrow{\cfcod} C_f
\xrightarrow{\cfcod} C_{(\cfcod : \bouquetGenL{B}{n} \to C_f)} \simeq \bouquetGenL{A}{n+1}
\]
%
where \( C_f \) denotes the cofibre of \( f \), \( \cfcod \) denotes the embedding
of the codomain into the cofibre, and the final equivalence is a standard characterisation
of $X_{n+1}/X_{n}$ for CW structures~\cite{BuchholtzFavonia18}.
%
This is a cofibre sequence, and so the following sequence is exact:
%
\[
\!\!\hredstr{n}\Big(\bouquetGen{A}{n}\Big) \xrightarrow{f_*} \hredstr{n}\Big(\bouquetGen{B}{n}\Big) \xrightarrow{\cfcod_*} \hredstr{n}{(C_f)} \to 0
\]
%
where the final $0$ comes from that fact that $\hredstr{n}$ vanishes on $\bouquetGen{A}{n+1}$.
We can compute the first two homology groups using additivity, and thus we see that
$\hredstr{n}(C_f) \cong \bZ[B]/\bZ[A]$.
%
Similarly, the group $\pi_n(X_\ast)$ is equal to \( \pi_n(X_{n+1}) \), which is computed
by the following lemma.

\begin{proposition}\label{prop:hurewicz-seq}
  For any $f : \bouquetGen{A}{n} \to \bouquetGen{B}{n}$ where $n\geq 1$ and $A$ and $B$ are finite
  types, there is an exact sequence
  \begin{center}
    \( \pi_n\left(\bouquetGen{A}{n}\right) \xrightarrow{f_*} \pi_n\left(\bouquetGen{B}{n}\right)
       \dhxrightarrow{\cfcod_*} \pi_n(C_f). \)
  \end{center}
\end{proposition}

\begin{proof}
  This follows from the Seifert-Van Kampen theorem \cite[Example 8.7.17]{HoTT13} in the case
  \( n = 1 \), and from the Blakers-Massey theorem~\cite{FavoniaFinster+16} in the case \( n > 1 \).
\end{proof}

The case \( n = 1 \) is a bit special: the fundamental group of a bouquet of circles is a
free group rather than a free abelian group. Therefore, in order to match what happens with homology,
we introduce the \emph{abelian homotopy group} functor \( \ab{n} \) which is simply defined as the
abelianisation of \( \pi_n \).
%
With this minor correction, we obtain that $\ab{n}(C_f) \cong \bZ[B]/\bZ[A]$.
%
Thus, we have proved that the \( n \)th homology group coincides with the abelianisation of the
\( n \)th homotopy group for a \( (n - 1) \)-connected CW complex.
%
The Hurewicz theorem actually goes a step further that this, and gives an explicit description
for the isomorphism between these two groups but checking that this map agrees with the one induced by our proof is direct.
%
\begin{theorem}
  Let $X$ be a CW complex.
  Define the \textbf{Hurewicz homomorphism} $\hur : \pi_n(X) \to \hredcw{n}(X)$ on canonical
  elements $f : \sphere{n} \to_\star X$ by letting $\hur(\trunc{f}) : \hredcw{n}(X)$ be the image
  of $1$ under the composition
  \( \bZ\ \raisemath{-1pt}{\xrightarrow{\sim}}\ \hredcw{n}(\sphere{n})
  \ \raisemath{-1pt}{\xrightarrow{f_*}}\ \hredcw{n}{(X)} \).
\end{theorem}

\begin{theorem}
  The Hurewicz homomorphism $\hur : \ab{n}(X) \to \hredcw{n}(X)$ is an isomorphism for any
  $(n-1)$-connected CW complex $X$.
\end{theorem}

\begin{proof}
  A formal proof is available at \url{https://github.com/loic-p/cellular/blob/main/Hurewicz/Map.agda}
  (\texttt{HurewiczTheorem}).
\end{proof}

In conclusion, we have a fully mechanised proof of the Hurewicz theorem for cellular homology.
%
This development was motivated by the recent proof of the Serre Finiteness theorem by Barton and
Campion~\cite{SerreFiniteness}, which relies on homology computations and the Hurewicz theorem.
%
We hope that our formal development will be helpful to the ongoing formalisation of the Serre
Finiteness theorem (by, in particular, Milner~\cite{MilnerTalk}).

\printbibliography


\end{document}
