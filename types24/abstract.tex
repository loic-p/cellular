% easychair.tex,v 3.5 2017/03/15

\documentclass{easychair}
%\documentclass[EPiC]{easychair}
%\documentclass[EPiCempty]{easychair}
%\documentclass[debug]{easychair}
%\documentclass[verbose]{easychair}
%\documentclass[notimes]{easychair}
%\documentclass[withtimes]{easychair}
%\documentclass[a4paper]{easychair}
%\documentclass[letterpaper]{easychair}

\usepackage{amsmath, amsfonts, amsthm , mathtools , bbold, float}

\usepackage{tikz}
\usepackage{tikz-cd}
\usepackage{mathpartir}
\usepackage[nameinlink,capitalize,noabbrev]{cleveref}
\usepackage{quiver}
\usepackage{xspace}
\usepackage{xcolor}
\usepackage{bbm}
\usepackage{breakurl}
\usepackage{bbding}
\hypersetup{pdfborder={0 0 100}}
\usepackage{lipsum}
%\usepackage{biblatex}
%\bibliographystyle{unsrt}
%% \usepackage[backend=biber,style=alphabetic,sorting=nty,giveninits]{biblatex} % Bibliography management
%% \addbibresource{refs.bib}

\usepackage{textcomp}
\usepackage{xspace}
\usepackage{url}
\usepackage{multirow}
\usepackage{hhline}
\usepackage{pifont}
\usetikzlibrary{arrows,matrix,decorations.pathmorphing,
  decorations.markings, calc, backgrounds}
\usepackage{mathpartir}
\usepackage{microtype}

\DisableLigatures[-]{family=tt*}

\definecolor{darkgreen}{rgb}{.30,.70,.60}

\newtheorem{remark}{Remark}
\newtheorem{lemma}{Lemma}
\newtheorem{definition}{Definition}
\newtheorem{example}{Example}
\newtheorem{proposition}{Proposition}
\newtheorem{theorem}{Theorem}
%\newtheorem{theoremCheck}[theorem]{Theorem${}^{\text{\color{darkgreen}{\Checkmark}}}$}
\newtheorem{corollary}{Corollary}
\newtheorem{conjecture}{Conjecture}


\newcommand{\push}[1]{{\ensuremath{\textnormal{\textsf{push}}}(#1)}}
\newcommand{\pushl}[1]{{\ensuremath{\textnormal{\textsf{push}}_{\textnormal{\textsf{l}}}}(#1)}}
\newcommand{\pushr}[1]{{\ensuremath{\textnormal{\textsf{push}}_{\textnormal{\textsf{r}}}}(#1)}}
\newcommand{\pushlr}{{\ensuremath{\textnormal{\textsf{push}}_{\textnormal{\textsf{lr}}}}}}
\newcommand{\smashinc}[2]{{\ensuremath{\langle #1,#2 \rangle}}}
\newcommand{\smashincalt}[3]{{\ensuremath{\langle #1,#2,#3 \rangle}}}

\newcommand{\incl}[1]{\textnormal{\ensuremath{\mathsf{incl}_{#1}}}}
\newcommand{\fin}[1]{\textnormal{\ensuremath{\mathsf{Fin}({#1})}}}


%% Configure appearance of system names here
\newcommand{\teletype}[1]{\ensuremath{\mathtt{#1}}}
\newcommand{\systemname}[1]{\teletype{\color{darkgray}#1}\xspace}

%% Systems referred to in the paper
\newcommand{\cubicaltt}{\systemname{cubicaltt}}
\newcommand{\cooltt}{\systemname{cooltt}}
\newcommand{\Idris}{\systemname{Idris}}
\newcommand{\Haskell}{\systemname{Haskell}}
\newcommand{\Lean}{\systemname{Lean}}

\definecolor{Revolutionary}{RGB}{232,70,68}
\newcommand{\redtt}{\textbf{\teletype{{\color{Revolutionary}red}tt}}}

\newcommand{\eg}{{e.g.}}
\newcommand{\ie}{{i.e.}}
\newcommand{\cf}{{cf.}}

%% Type theory macros
\newcommand{\su}[2]{#1/#2}
\newcommand{\subst}[2]{(\su #1 #2)}
\newcommand{\esubst}[3]{(#1, \su #2 #3)}
\newcommand{\substnop}[2]{{#2}\, / \,{#1}}

%% \usepackage{MnSymbol}
\usepackage{newunicodechar}
\newunicodechar{≃}{\ensuremath{\simeq}}
\newunicodechar{≅}{\ensuremath{\mathsfg}}
\newunicodechar{∈}{\ensuremath{\in}}
\newunicodechar{⁻}{\ensuremath{^{-}}}
\newunicodechar{ᴹ}{\ensuremath{_{M}}}
\newunicodechar{ᴺ}{\ensuremath{_{N}}}
\newunicodechar{≡}{\ensuremath{\equiv}}
\newunicodechar{λ}{\ensuremath{\lambda}}
\newunicodechar{⊎}{\ensuremath{\uplus}}
\newunicodechar{∷}{\ensuremath{::}}
\newunicodechar{ℓ}{\ensuremath{\ell}}
\newunicodechar{ᵢ}{\ensuremath{_i}}
\newunicodechar{⟨}{\ensuremath{\langle}}
\newunicodechar{⟩}{\ensuremath{\rangle}}
\newunicodechar{α}{\ensuremath{\alpha}}
\newunicodechar{β}{\ensuremath{\beta}}
\newunicodechar{θ}{\ensuremath{\theta}}
\newunicodechar{φ}{\ensuremath{\varphi}}
\newunicodechar{ψ}{\ensuremath{\psi}}
\newunicodechar{η}{\ensuremath{\eta}}
\newunicodechar{ε}{\ensuremath{\varepsilon}}
\newunicodechar{ι}{\ensuremath{\iota}}
\newunicodechar{Σ}{\ensuremath{\Sigma}}
\newunicodechar{σ}{\ensuremath{\sigma}}
\newunicodechar{∀}{\ensuremath{\forall}}
\newunicodechar{ℕ}{\ensuremath{\mathbb{N}}}
\newunicodechar{→}{\ensuremath{\to}}
\newunicodechar{⊎}{\ensuremath{\uplus}}
\newunicodechar{⋆}{\ensuremath{*}}
\newunicodechar{¬}{\ensuremath{\lnot}}
\newunicodechar{Θ}{\ensuremath{\theta}}
\newunicodechar{ᴸ}{\ensuremath{{}^{\color{black}\leftarrow}}}  % hax
\newunicodechar{ᴿ}{\ensuremath{{}^{\color{black}\rightarrow}}} % hax
\newunicodechar{∧}{\ensuremath{\wedge}}
\newunicodechar{∨}{\ensuremath{\vee}}
\newunicodechar{∼}{\ensuremath{\sim}}
\newunicodechar{≢}{\ensuremath{\nequiv}}
\newunicodechar{Π}{\ensuremath{\Pi}}
\newunicodechar{ℤ}{\ensuremath{\mathbb{Z}}}
\newunicodechar{∥}{\ensuremath{\parallel}}
\newunicodechar{∣}{\ensuremath{\mid}}
\newunicodechar{ℚ}{\ensuremath{\mathbb{Q}}}
\newunicodechar{₊}{\ensuremath{_+}}
\newunicodechar{∗}{\ensuremath{\ast}}
\newunicodechar{₁}{\ensuremath{_1}}
\newunicodechar{₂}{\ensuremath{_2}}
\newunicodechar{⊥}{\ensuremath{\bot}}
\newunicodechar{∘}{\ensuremath{\circ}}
\newunicodechar{ρ}{\ensuremath{\rho}}
\newunicodechar{↦}{\ensuremath{\mapsto}}
\newunicodechar{₀}{\ensuremath{_0}}
\newunicodechar{∙}{\ensuremath{\boldsymbol{\cdot}}}
\newunicodechar{Ω}{\ensuremath{\Omega}}
\newunicodechar{§}{\ensuremath{\mathbb{S}}}
\newunicodechar{𝟙}{\ensuremath{\mathbb{1}}}
\newunicodechar{×}{\ensuremath{\times}}

%Ugly hacks for torus, klein and RP2
\newunicodechar{□}{\mathsf{\ensuremath{\square}}}
\newunicodechar{α}{\textsf{\ensuremath{{\mathbb{T}^2}}}} %Torus

\newunicodechar{β}{\textsf{\ensuremath{{\mathbb{K}^2}}}} %Klein

\newunicodechar{γ}{\textsf{\ensuremath{{\mathbb{R}P^2}}}} %RP^2


% Hacks for ex computation
\newunicodechar{ₕ}{\textsf{\ensuremath{_h}}}
\newunicodechar{ϕ}{\textsf{\ensuremath{\phi}}}
\newunicodechar{ζ}{\anum{3}\;\;\;\;\;\;\;\;\;}

\newcommand{\winding}[1]{\textsf{winding}({#1})}
\newcommand{\Type}{\textsf{Type}}
\newcommand{\ptrunc}[1]{\textsf{∥}\,{#1}\,\textsf{∥}}
\newcommand{\tyProduct}[2]{{#1}\,\AgdaOperator{\AgdaTextsftion{×}}\,{#2}}
\newcommand{\tySigma}[3]{\textsf{Σ[}\,{#1}\,\textsf{∈}\,{#2}\,\textsf{]}\,{#3}}
\newcommand{\tyMaybe}[1]{\data{Maybe}\,{#1}}
\newcommand{\tyNat}{\data{ℕ}}
\newcommand{\tyPath}[2]{{#1}\,\textsf{≡}\,{#2}}
\newcommand{\tyPathP}[4]{\primty{PathP}\,(\symb{λ}\,{#1} \to {#2})\,{#3}\,{#4}}
\newcommand{\tyEquiv}[2]{{#1}\,\textsf{≃}\,{#2}}
\newcommand{\tySum}[2]{{#1}\,\textsf{⊎}\,{#2}}
\newcommand{\tyNot}[1]{\textsf{¬}\,{#1}}
\newcommand{\tyStructEq}[3]{{#1}\,\textsf{≃[}\,{#2}\,\textsf{]}\,{#3}}



\newlength{\LETTERheight}
\AtBeginDocument{\settoheight{\LETTERheight}{I}}
\newcommand*{\longleadsto}[1]{\ \raisebox{0.24\LETTERheight}{\tikz \draw [->,
line join=round,
decorate, decoration={
    zigzag,
    segment length=4,
    amplitude=.9,
    post=lineto,
    post length=2pt
}] (0,0) -- (#1,0);}\ }

\newcommand{\cmark}{\ding{51}}%
\newcommand{\xmark}{\ding{55}}%
\newcommand{\equivDef}{\overset{\scriptscriptstyle \textsf{def}}{\equiv}}
\newcommand{\trivGrp}{\textsf{𝟙}}
\newcommand{\bZ}{\textsf{ℤ}}
\newcommand{\bN}{\textsf{$\mathbb{N}$}}
\newcommand{\sphere}[1]{\textsf{§}^{#1}}
\newcommand{\klein}{\textsf{$\mathbb{K}^{\,2}$}}
\newcommand{\cohom}[2]{H^{#1}\!\left({#2}\right)}
\newcommand{\funspace}[2]{\left({{#2} \to \textsf{K}_{#1}}\right)}
\newcommand{\funspaceP}[2]{{{#2} \to \textsf{K}_{#1}}}
\newcommand{\torus}{\textsf{$\mathbb{T}^{\,2}$}}
\newcommand{\RP}{\textsf{$\mathbb{R}P^2$}}
\newcommand{\RPinf}{\textnormal{\ensuremath{\mathbb{R}P^\infty}}}
\newcommand{\CP}{\textsf{$\mathbb{C}P^2$}}
\newcommand{\refl}{\textsf{refl}}
\newcommand{\inr}[1]{\mathsf{inr}\,{#1}}
\newcommand{\inl}[1]{\mathsf{inl}\,{#1}}
\newcommand{\trunc}[1]{\mathsf{∣}\,#1\,\mathsf{∣}}
\newcommand{\truncT}[2]{\textsf{∥}\,#2\,\textsf{∥}_{#1}}
\newcommand{\Loop}{\mathsf{loop}}
\newcommand{\base}{\mathsf{base}}
\newcommand{\north}{\mathsf{north}}
\newcommand{\south}{\mathsf{south}}
\newcommand{\Code}{\textsf{Code}}
\newcommand{\merid}[1]{\mathsf{merid}\;{#1}}
\newcommand{\transport}[2]{\textsf{transport}\;(#1)\;#2}
\newcommand{\ap}[2]{\textnormal{\textsf{ap}}_{#1}{(#2)}}
\newcommand{\app}[3]{\textnormal{\textsf{ap}$^2$}_{#1}\;{#2}\;{#3}}
\newcommand{\tmPair}[2]{{#1}\,\mathsf{,}\,{#2}}
\newcommand{\tmTrunc}[1]{\mathsf{[}\,{#1}\,\mathsf{]}}
\newcommand{\tmNil}{\mathsf{[]}}
\newcommand{\tmCons}[2]{{#1}\;\mathsf{∷}\;{#2}}

\newcommand{\cupprodop}{\textsf{$\smile$}}
\newcommand{\cupprodkop}{\textsf{$\smile_k$}}
\newcommand{\cupprodk}[2]{{#1}\;\cupprodkop\;{#2}}
\newcommand{\cupprod}[2]{{#1}\;\cupprodop\;{#2}}

%% Comment macro colors
\definecolor{dkblue}{rgb}{0,0.1,0.5}
\definecolor{lightblue}{rgb}{0,0.5,0.5}
\definecolor{dkgreen}{rgb}{0,0.6,0}
\definecolor{dkbrown}{rgb}{0.4,0,0}
\definecolor{dkviolet}{rgb}{0.6,0,0.8}

%% SUBMISSION: turn off all comments
%\newcommand{\mycomment}[3]{}
\newcommand{\mycomment}[3]{\textcolor{#1}{[#2#3]}}
\newcommand{\todo}[1]{\mycomment{red}{TODO: }{#1}}

\newcommand{\sq}[1]{\textnormal{\ensuremath{\mathsf{Sq}^{#1}}}}
\newcommand{\sqind}[2]{\textnormal{\ensuremath{\mathsf{Sq}_{#2}^{#1}}}}

\DeclareMathOperator*{\bigast}{\raisebox{-0.6ex}{\scalebox{2.5}{$\ast$}}}
\DeclareMathOperator*{\bigsmile}{\raisebox{-0.ex}{\scalebox{1.5}{$\smile$}}}

\newcommand{\negpath}{\textnormal{\textsf{neg-path}}}
\newcommand{\totSq}{\textnormal{\ensuremath{\widehat{\mathsf{Sq}}}}}

\newcommand{\cwskel}{\textnormal{\ensuremath{\mathsf{CW}^{\mathsf{skel}}}}}
\newcommand{\cwskelinf}{\textnormal{\ensuremath{\mathsf{CW}_\infty^{\mathsf{skel}}}}}
\newcommand{\hskel}[1]{\textnormal{\ensuremath{H^{\mathsf{skel}}_{#1}}}}
\newcommand{\hskelinf}[1]{\textnormal{\ensuremath{H^{\mathsf{skel}_\infty}_{#1}}}}
\newcommand{\abgrp}{\textnormal{\ensuremath{\mathsf{AbGrp}}}}



\usepackage{doc}

% use this if you have a long article and want to create an index
% \usepackage{makeidx}

% In order to save space or manage large tables or figures in a
% landcape-like text, you can use the rotating and pdflscape
% packages. Uncomment the desired from the below.
%
% \usepackage{rotating}
% \usepackage{pdflscape}

% Some of our commands for this guide.
%
\newcommand{\easychair}{\textsf{easychair}}
\newcommand{\miktex}{MiK{\TeX}}
\newcommand{\texniccenter}{{\TeX}nicCenter}
\newcommand{\makefile}{\texttt{Makefile}}
\newcommand{\latexeditor}{LEd}

%\makeindex

%% Front Matter
%%
% Regular title as in the article class.
%
\title{A Constructive Cellular Approximation Theorem in HoTT}

% Authors are joined by \and. Their affiliations are given by \inst, which indexes
% into the list defined using \institute
%
\author{
Axel Ljungström \and Loïc Pujet}

% Institutes for affiliations are also joined by \and,
\institute{
  {
  Stockholm University,
  Stockholm, Sweden
  }
 }

%  \authorrunning{} has to be set for the shorter version of the authors' names;
% otherwise a warning will be rendered in the running heads. When processed by
% EasyChair, this command is mandatory: a document without \authorrunning
% will be rejected by EasyChair

\authorrunning{Axel Ljungström and Loïc Pujet}

% \titlerunning{} has to be set to either the main title or its shorter
% version for the running heads. When processed by
% EasyChair, this command is mandatory: a document without \titlerunning
% will be rejected by EasyChair
\titlerunning{A Constructive Cellular Approximation Theorem in HoTT}


\begin{document}

\maketitle

At the heart of Homotopy Type Theory (HoTT) is the analogy between types and spaces.
%
This analogy permits the use of type theory as a language for Algebraic Topology, \ie for the study
of spaces and maps of spaces up to homotopy by means of algebraic invariants, such as homotopy groups
and homology groups~\cite{HoTT13}.
%
Although the methods of algebraic topology apply to very general notions of spaces, the theory is
most easily developed in the context of a more restricted and well-behaved notion of spaces:
CW complexes.
%
As such, it is natural to define CW complexes in the language of Homotopy Type Theory,
in order to obtain a notion of space that is easier to work with than arbitrary types.

In this work, we revisit the definition of CW complexes that was given by Favonia and
Buchholtz~\cite{BuchholtzFavonia18} and we develop their theory.
%
In particular, we focus on the \emph{cellular approximation theorem}, a cornerstone result in
Algebraic Topology which says that arbitrary maps between CW complexes and their homotopies may be
approximated by maps and homotopies that respect the cellular structure~\cite[chap. 10]{May1999}.
%
We give a constructive proof of the theorem that relies heavily on the (admissible) principle of
finite choice%
\footnote{The proof has been fully formalised in \CubicalAgda, and is available at \url{https://github.com/loic-p/cellular/blob/main/summary.agda}}%
, and we discuss the extent to which the theorem can be strengthened while remaining constructive.

We will need the following definition to define CW complexes:

\begin{definition}[CW skeleta]
  An \textbf{ordered CW skeleton} is an infinite sequence of types $$C_{-1} \xrightarrow{\iota_{-1}} C_0 \xrightarrow{\iota_{0}} C_1 \xrightarrow{\iota_{1}} \dots$$
  equipped with maps $\alpha : \sphere{n} \times A_n \to C_n$ where $A_n$ is equivalent to $\mathsf{Fin}(k_n)$ for some $k_n : \mathbb{N}$ and the following square is a pushout:
  \[
\begin{tikzcd}[ampersand replacement=\&]
	{\sphere{n} \times A_n} \& {A_n} \\
	{C_n} \& {C_{n+1}}
	\arrow[from=1-1, to=1-2]
	\arrow["{\alpha_n}"', from=1-1, to=2-1]
	\arrow["{\iota_n}"', from=2-1, to=2-2]
	\arrow[from=1-2, to=2-2]
	\arrow["\lrcorner"{anchor=center, pos=0.125, rotate=180}, draw=none, from=2-2, to=1-1]
\end{tikzcd}
\]
An \textbf{unordered CW skeleton} is defined similarly, but each $A_n$ is only assumed to be merely
finite, \ie for all \( n \) we have a proof of \( \| A_n \simeq \mathsf{Fin}(k_n) \|_{-1} \).
\end{definition}

The pushout condition ensures that the \( (n + 1) \)-skeleton \( C_{n+1} \) is obtained by attaching
a finite number of \( n \)-dimensional cells to the \( n \)-skeleton \( C_n \). In the case of an
ordered CW skeleton, the cells are equipped with an order inherited from \( \mathsf{Fin}(k_n) \),
hence the name.
%
Given a CW complex \( C_{\bullet} \), we write \( C_\infty \) for the colimit of the sequence of
\( n \)-skeleta, and for any \( n \) we write $\iota_\infty : C_n \to C_\infty$ for the inclusion of
$C_n$ into the colimit $C_\infty$.

\begin{definition}[CW complexes]
  A type \( X \) is said to be an ordered (resp. unordered) CW complex if there merely exists an
  ordered (resp. unordered) CW skeleton \( C_{\bullet} \) such that \( X \) is equivalent to the
  colimit \( C_\infty \).
\end{definition}

A map between two CW complexes \( X \) and \( Y \) is simply a map between the underlying types.
%
The cellular approximation theorem states that such maps may be approximated by \emph{cellular maps},
\ie sequences of maps between the \( n \)-skeleta of \( X \) and \( Y \).
%
In order to prove this theorem constructively for unordered CW complexes, we need to define finite
cellular maps:

\begin{definition}[Cellular $m$-maps]
  Given two CW skeleta $C_\bullet$ and $D_\bullet$, a cellular $m$-map from \( C_\bullet \) to \( D_\bullet \)
  is a finite sequence of maps $(f_n : C_n \to D_n)_{n \leq m}$ equipped with a family of homotopies
  $h_n : \iota_n \circ f_{n} = f_{n+1} \circ \iota_n$ for $n < m$.
\end{definition}

\begin{definition}[Cellular $m$-homotopies]
  Given two CW skeleta $C_\bullet,D_\bullet$ and two cellular $m$-maps $f_\bullet , g_\bullet : C_\bullet \to D_\bullet$, a
  $m$-homotopy between \( f_\bullet \) and \( g_\bullet \) is a finite sequence of homotopies
  \( (h_n : \iota_n \circ f_n = \iota_n \circ g_n)_{n \leq m} \) such that for all \( n < m \) and for all \( x \) the following square commutes:
  \[
\begin{tikzcd}[ampersand replacement=\&]
	{(\iota_{n+1} \circ f_{n+1} \circ \iota_n)~x} \& {(\iota_{n+1} \circ g_{n+1} \circ \iota_n)~x} \\
	{(\iota_{n+1} \circ \iota_n \circ f_n)~x} \& {(\iota_{n+1} \circ \iota_n \circ g_n)~x}
	\arrow[double equal sign distance, -, double, "{h_{n+1} \circ \iota_n}"', from=1-1, to=1-2]
	\arrow[double equal sign distance, -, double, from=1-1, to=2-1]
	\arrow[double equal sign distance, -, double, "{\iota_{n+1} \circ h_n}"', from=2-1, to=2-2]
	\arrow[double equal sign distance, -, double, from=1-2, to=2-2]
\end{tikzcd}
\]
\end{definition}

\begin{theorem}[Cellular $m$-approximation theorem]
  Given two unordered CW skeleta $C_\bullet,D_\bullet$, a map $f : C_\infty \to  D_\infty$ and
  $m:\bN$, there merely exists a $m$-cellular map $f_\bullet : C_\bullet \to D_{\bullet}$ such that
  $\iota_{\infty} \circ f_m = f \circ \iota_{\infty}$.
\end{theorem}

\begin{theorem}[Cellular $m$-approximation theorem, part 2]
  Let $C_\bullet,D_\bullet$ be unordered CW skeleta and consider two
  cellular $m$-maps map $f_\bullet,g_\bullet : C_\bullet \to D_\bullet$ with a
  homotopy $h_m : f_m = g_m$. In this case, there merely exists a
  cellular $m$-homotopy $f_\bullet \sim_m g_\bullet$.
\end{theorem}

\begin{proof}[Sketch of proof]
The proof of theorems 1 and 2 is done by induction on \( m \):
%
if we have a \( n \)-approximation of \( f \), we can use the principle of finite choice to obtain
the mere existence of a \( (n+1) \)-approximation.
%
Note that we only approximate \( f \) up to a fixed dimension \( m \), so that the construction
only needs finitely many calls to finite choice, which is constructively valid~\cite[exercise 3.22]{HoTT13}.
\end{proof}

Our statement of the cellular approximation theorem is weaker than its classical counterpart on two
points.
%
Firstly, we only obtain the \emph{mere existence} of an approximation. However, since every
construction in HoTT has to be homotopy invariant, the untruncated statement is actually
inconsistent: when specialised to the unit type and the circle (which are both CW complexes),
the untruncated approximation theorem amounts to the contractibility of the circle.
Therefore, some amount of truncation is required to state the theorem in HoTT.
%
Secondly, the classical cellular approximation theorems is stated for \( m = \infty \), while
ours only provides finite approximations.
%
In fact, due to the fundamental reliance of the theorem on finite choice, we conjecture that the
case \( m = \infty \) is equivalent to the axiom of countable choice, and thus not provable in
constructive HoTT.

\begin{conjecture}
  The case \( m = \infty \) of the cellular approximation theorems is not provable in plain HoTT
  for unordered CW skeleta.
\end{conjecture}

%% Indeed, we expect that it is equivalent to the possibility of choosing an inhabitant of countably
%% many types that are merely finite.
%
However, in the case of \emph{ordered} CW skeleta, we expect that it is possible to use the order
on the sets of cells to pick a \emph{minimal} approximation at each stage for some carefully defined
order, which eschews the need for finite choice.
%
Hence the following conjecture:

\begin{conjecture}\label{conj1}
  The cellular approximation theorems hold for $m = \infty$ for ordered CW skeleta.
\end{conjecture}


\label{sect:bib}
\bibliographystyle{plain}
%\bibliographystyle{alpha}
%\bibliographystyle{unsrt}
%\bibliographystyle{abbrv}
\bibliography{refs.bib}

%------------------------------------------------------------------------------
% Index
%\printindex

%------------------------------------------------------------------------------
\end{document}
